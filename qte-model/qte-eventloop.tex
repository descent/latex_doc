\chapter{QT/E event loop {\MbQ\cH188}\z{\McQ\cH6}}
QApplication::exec() {\MbQ\cH125}\z{\MdQ\cH227}\z{\MdQ\cH202} QApplication::eventLoop(),
{\McQ\cH55} QApplication::eventLoop() {\MbQ\cH125}\z{\MbQ\cH223}\z{\MbQ\cH222}\z{\McQ\cH248}\z{\MaQ\cH95} QEventLoop {\MbQ\cH211}\z{\MaQ\cH75},
{\MbQ\cH205}\z{\MbQ\cH40}\z{\MaQ\cH202}\z{\MaQ\cH207}\z{\McQ\cH87} QEventLoop::exec(){\MaQ\cH1}\zZ
{\MaQ\cH74}\z{\MaQ\cH45}\z{\MbQ\cH117} source code{\MaQ\cH1}\zZ

\begin{lstlisting}[tabsize=1,basicstyle=\footnotesize,frame=single,frameround=tttt,numbers=left,breaklines=true,title=\$QTDIR/src/kernel/qapplication.cpp,escapeinside={/*@}{@*/}]
int QApplication::exec()
{
 return eventLoop()->exec();
}

QEventLoop *QApplication::eventLoop()
{
 if ( !eventloop && !is_app_closing )
  (void) new QEventLoop( qApp, "default event loop" );
 return eventloop;
}
\end{lstlisting}


\begin{lstlisting}[tabsize=1,basicstyle=\footnotesize,frame=single,frameround=tttt,numbers=left,breaklines=true,title=\$QTDIR/src/kernel/qeventloop.cpp,escapeinside={/*@}{@*/}]
int QEventLoop::exec()
{
 d->reset();

 /*@\colorbox{slight}{enterLoop();}@*/

 // cleanup
 d->looplevel = 0;
 d->quitnow  = FALSE;
 d->exitloop = FALSE;
 d->shortcut = FALSE;
 // don't reset quitcode!

 return d->quitcode;
}


int QEventLoop::enterLoop()
{
 // save the current exitloop state
 bool old_exitloop = d->exitloop;
 d->exitloop = FALSE;
 d->shortcut = FALSE;

 d->looplevel++;
 while ( ! d->exitloop )
  /*@\label{processEvents} \colorbox{slight}{processEvents( AllEvents  $|$  WaitForMore );}@*/
 d->looplevel--;

 // restore the exitloop state, but if quitnow is TRUE, we need 
 // to keep exitloop set so that all other event loops drop out.
 d->exitloop = old_exitloop || d->quitnow;
 d->shortcut = d->quitnow;

 if ( d->looplevel < 1 ) 
 {
  d->quitnow  = FALSE;
  d->exitloop = FALSE;
  d->shortcut = FALSE;
  emit qApp->aboutToQuit();

  // send deferred deletes
  /*@\label{sendPostedEvents}\colorbox{slight}{QApplication::sendPostedEvents( 0, QEvent::DeferredDelete );}@*/
 }

 return d->looplevel;
}

\end{lstlisting}


QEventLoop::exec() {\MbQ\cH125}\z{\MdQ\cH227}\z{\MdQ\cH202} QEventLoop::enterLoop(){\MaQ\cH1}\zZ
QEventLoop::enterLoop() {\MbQ\cH127}\z{\MaQ\cH115}\z{\MaQ\cH95}\z{\MbQ\cH170}\z{\McQ\cH156}\z{\McQ\cH189}\z{\McQ\cH98}\z{\MbQ\cH237} function call:

\begin{enumerate}
\item QEventLoop::processEvents(ProcessEventsFlags flags) ({\MaQ\cH202} qeventloop\_{}qws.cpp)

\begin{lstlisting}[tabsize=1,basicstyle=\footnotesize,frame=single,frameround=tttt,numbers=left,breaklines=true,title=\$QTDIR/src/kernel/qeventloop\_{}qws.cpp,escapeinside={/*@}{@*/}]
bool QEventLoop::processEvents( ProcessEventsFlags flags )
{
    // process events from the QWS server
    int	   nevents = 0;

#if defined(QT_THREAD_SUPPORT)
    QMutexLocker locker( QApplication::qt_mutex );
#endif

    // handle gui and posted events
    if (qt_is_gui_used ) {
	QApplication::sendPostedEvents();

	while ( qt_fbdpy->eventPending() ) {	// also flushes output buffer
	    if ( d->shortcut ) {
		return FALSE;
	    }

	    QWSEvent *event = qt_fbdpy->getEvent();	// get next event
	    nevents++;

	    bool ret = qApp->qwsProcessEvent( event ) == 1;
	    delete event;
	    if ( ret ) {
		return TRUE;
	    }
	}
    }

    if ( d->shortcut ) {
	return FALSE;
    }

    extern QPtrQueue<QWSCommand> *qt_get_server_queue();
    if ( !qt_get_server_queue()->isEmpty() ) {
	QWSServer::processEventQueue();
    }

    QApplication::sendPostedEvents();

    // don't block if exitLoop() or exit()/quit() has been called.
    bool canWait = d->exitloop || d->quitnow ? FALSE : (flags & WaitForMore);

    // Process timers and socket notifiers - the common UNIX stuff

    // return the maximum time we can wait for an event.
    static timeval zerotm;
    timeval *tm = qt_wait_timer();		// wait for timer or event
    if ( !canWait ) {
	if ( !tm )
	    tm = &zerotm;
	tm->tv_sec  = 0;			// no time to wait
	tm->tv_usec = 0;
    }

    int highest = 0;
    if ( ! ( flags & ExcludeSocketNotifiers ) ) {
	// return the highest fd we can wait for input on
	if ( d->sn_highest >= 0 ) {                     // has socket notifier(s)
	    if ( d->sn_vec[0].list && ! d->sn_vec[0].list->isEmpty() )
		d->sn_vec[0].select_fds = d->sn_vec[0].enabled_fds;
	    else
		FD_ZERO( &d->sn_vec[0].select_fds );

	    if ( d->sn_vec[1].list && ! d->sn_vec[1].list->isEmpty() )
		d->sn_vec[1].select_fds = d->sn_vec[1].enabled_fds;
	    else
		FD_ZERO( &d->sn_vec[1].select_fds );

	    if ( d->sn_vec[2].list && ! d->sn_vec[2].list->isEmpty() )
		d->sn_vec[2].select_fds = d->sn_vec[2].enabled_fds;
	    else
		FD_ZERO( &d->sn_vec[2].select_fds );
	} else {
	    FD_ZERO( &d->sn_vec[0].select_fds );
	    FD_ZERO( &d->sn_vec[1].select_fds );
	    FD_ZERO( &d->sn_vec[2].select_fds );
	}

	highest = d->sn_highest;
    } else {
	FD_ZERO( &d->sn_vec[0].select_fds );
	FD_ZERO( &d->sn_vec[1].select_fds );
	FD_ZERO( &d->sn_vec[2].select_fds );
    }

    FD_SET( d->thread_pipe[0], &d->sn_vec[0].select_fds );
    highest = QMAX( highest, d->thread_pipe[0] );

    if ( canWait )
	emit aboutToBlock();

    if ( qt_preselect_handler ) {
	QVFuncList::Iterator it, end = qt_preselect_handler->end();
	for ( it = qt_preselect_handler->begin(); it != end; ++it )
	    (**it)();
    }

    // unlock the GUI mutex and select.  when we return from this function, there is
    // something for us to do
#if defined(QT_THREAD_SUPPORT)
    locker.mutex()->unlock();
#endif

    int nsel;
    nsel = select( highest + 1,
		   &d->sn_vec[0].select_fds,
		   &d->sn_vec[1].select_fds,
		   &d->sn_vec[2].select_fds,
		   tm );

    // relock the GUI mutex before processing any pending events
#if defined(QT_THREAD_SUPPORT)
    locker.mutex()->lock();
#endif

    // we are awake, broadcast it
    emit awake();
    emit qApp->guiThreadAwake();

    if ( nsel == -1 ) {
	if ( errno != EINTR && errno != EAGAIN )
	    perror( "select" );
	return (nevents > 0);
    }

#undef FDCAST

    // some other thread woke us up... consume the data on the thread pipe so that
    // select doesn't immediately return next time
    if ( nsel > 0 && FD_ISSET( d->thread_pipe[0], &d->sn_vec[0].select_fds ) ) {
	char c;
	::read( d->thread_pipe[0], &c, 1 );
    }

    if ( qt_postselect_handler ) {
	QVFuncList::Iterator it, end = qt_postselect_handler->end();
	for ( it = qt_postselect_handler->begin(); it != end; ++it )
	    (**it)();
    }

    // activate socket notifiers
    if ( ! ( flags & ExcludeSocketNotifiers ) && nsel > 0 && d->sn_highest >= 0 ) {
	// if select says data is ready on any socket, then set the socket notifier
	// to pending
	int i;
	for ( i=0; i<3; i++ ) {
	    if ( ! d->sn_vec[i].list )
		continue;

	    QPtrList<QSockNot> *list = d->sn_vec[i].list;
	    QSockNot *sn = list->first();
	    while ( sn ) {
		if ( FD_ISSET( sn->fd, &d->sn_vec[i].select_fds ) )
		    setSocketNotifierPending( sn->obj );
		sn = list->next();
	    }
	}

	nevents += activateSocketNotifiers();
    }

    // activate timers
    nevents += activateTimers();

    // return true if we handled events, false otherwise
    return (nevents > 0);
}


\end{lstlisting}

QEventLoop::processEvents() {\MbQ\cH127}\z{\MaQ\cH115}\z{\MaQ\cH95}\z{\MhQ\cH23}\z{\MbQ\cH133}, {\McQ\cH248}\z{\MaQ\cH95}\z{\MbQ\cH117}
\begin{verbatim}
void processEvents( ProcessEventsFlags flags, int maxtime );
virtual bool processEvents( ProcessEventsFlags flags );
\end{verbatim}
{\McQ\cH165}\z{\MiQ\cH179}\z{\MdQ\cH227}\z{\MdQ\cH202}\z{\MbQ\cH237}\z{\MbQ\cH117} virtual bool processEvents( ProcessEventsFlags flags )
(in qeventloop\_{}qwqs.cpp){\MaQ\cH1}\zZ
{\MbQ\cH164} member function {\MbQ\cH127}\z{\MaQ\cH85}\z{\MbQ\cH224}\z{\McQ\cH248}\z{\MaQ\cH95} qt\_{}fbdpy, {\MeQ\cH133}\z{\MbQ\cH237} type {\MbQ\cH209} QWSDisplay{\MaQ\cH1}\zZ

\item QApplication::sendPostedEvents()\footnote{{\MaQ\cH223}\z{\MdQ\cH100}\z{\MbQ\cH39}\z{\McQ\cH189}\z{\McQ\cH98}\z{\MbQ\cH237}\z{\MbQ\cH154}\z{\MaQ\cH228}, source code
{\MbQ\cH245}\z{\McQ\cH150}\z{\MaQ\cH86}\z{\MbQ\cH39}\z{\MiQ\cH181}\z{\MjQ\cH198}{\MaQ\cH1}}
({\MaQ\cH202} qapplication.cpp)
\end{enumerate}


