\documentclass[12pt]{article}
%\usepackage[paperwidth=8,paperheight=11cm,textwidth=13cm,textheight=14cm,left=0.5cm,right=0.5cm,top=0.5cm,nohead,nofoot]{geometry} 


\usepackage[paperwidth=10.1cm,paperheight=13cm,left=0.3cm,right=0.3cm,bottom=0.5cm,top=0.1cm,nohead,nofoot]{geometry} 

%\usepackage[papersize={4.5in,6in},margin=0.5cm]{geometry}
%\usepackage[body={2cm, 5.5in},left=0.2in,right=0.2in,top=0.2in,bottom=0.2in]{geometry}
\usepackage{multicol}

\parindent=0cm
\parskip=13pt
% 轉成 PDF 時, 產生 hyperlink 的效果
%\usepackage{CJK}
\usepackage{CJKutf8} %使用 CJKutf8 套件, 配合, \hypersetup 產生中文 pdf 書籤
%\usepackage[dvips]{hyperref}
\usepackage{type1cm}
%\usepackage{fancyvrb}
\usepackage{graphicx}
\usepackage{lastpage}
\usepackage[dvips]{color}
\usepackage{listings}
\usepackage{comment}
%\usepackage{ucs}
%\usepackage[utf8]{inputenc}
% 在使用 latex2html 可以處理中文
%\begin{htmlonly}
%\usepackage{taiwan}
%\end{htmlonly}
%\usepackage{cwtex}
%\begin{htmlonly}
%\usepackage{html}
%\end{htmlonly}
\usepackage{appendix}

%\usepackage[dvips,CJKbookmarks,pdftitle={From TeX/LaTeX to PDF}%
%                ,pdfsubject={From TeX/LaTeX to PDF}%
%		 ,pdfkeywords={tex,latex,cjk,typeset,pdf,typography}]{hyperref}
\usepackage{hyperref} % 最好保证 hyperref 是最后加载的宏包
\hypersetup{%
  dvipdfmx,% 设定要使用的 driver 为 dvipdfmx
  unicode={true},% 使用 unicode 来编码 PDF 字符串
  pdfstartview={FitH},% 文档初始视图为匹配宽度
  bookmarksnumbered={true},% 书签附上章节编号
  bookmarksopen={true},% 展开书签
  pdfborder={0 0 0},% 链接无框
  pdftitle={Coroutines in C}
  pdfauthor={Simon Tatham}
  pdfsubject={Coroutines in C}
  pdfkeywords={coroutines},
  pdfcreator={应用程序},
  pdfproducer={PDF 制作程序},% 这个好像没起作用?
}

%\begin{htmlonly}

%\renewcommand{\abstractname}{\r18 摘要}
%\renewcommand{\figurename}{\m11 圖}
%\renewcommand{\contentsname}{\r20 目錄}

%\end{htmlonly}

%\renewcommand{\appendixpagenam}{附錄}

% 標題頁
% 正文
\title{Coroutines in C}
\author{Simon Tatham}
\begin{document}
\maketitle
%\begin{CJK}{UTF8}{cwku}
%\renewcommand{\contentsname}{目錄}

%\begin{htmlonly}
%\include{auth}
%\end{htmlonly}

\newpage
%\addtocontents{toc}{text}
\tableofcontents
%\addcontentsline{toc}{section}{abstract}
%\addcontentsline{toc}{section}{REFERENCE}
\newpage

\begin{CJK}{UTF8}{bsmi} %開始CJK環境, 設定編碼, 設定字體

\section{Introduction}

 Structuring a large program is always a difficult job. One of the particular problems that often comes up is this: if you have a piece of code producing data, and another piece of code consuming it, which should be the caller and which should be the callee?

我們都知道構建一個大型的程序是一件非常困難的工作。一個經常面臨的問題來自於此:如果你有一個程式碼片段用來生產數據, 另一個程式碼片段來使用它, 哪一個是呼叫者, 哪一個是被呼叫者呢?

 Here is a very simple piece of run-length decompression code, and an equally simple piece of parser code: 


這裡有一個簡單的解壓程式碼片段和一個同樣簡單的解析程式碼片段: 

\newpage
\begin{multicols}{2}

\begin{lstlisting}[caption=decompression, basicstyle=\footnotesize, breaklines=true, frame=single,frameround=tttt]
//Decompression code
while (1) 
{
  c = getchar();

  if (c == EOF)
    break;
  if (c == 0xFF) 
  {
    len = getchar();
    c = getchar();
    while (len--)
      emit(c);
  } 
  else
    emit(c);
}
emit(EOF);
\end{lstlisting}

\begin{lstlisting}[caption=parser, basicstyle=\footnotesize, breaklines=true, frame=single,frameround=tttt]
// Parser code 
while (1) 
{
 c = getchar();
 if (c == EOF)
    break;
 if (isalpha(c)) 
 {
  do 
  {
   add_to_token(c);
   c = getchar();
  }while(isalpha(c));
  got_token(WORD);
 }
 add_to_token(c);
 got_token(PUNCT);
}

\end{lstlisting}
\end{multicols}

 Each of these code fragments is very simple, and easy to read and understand. One produces a character at a time by calling emit(); the other consumes a character at a time by calling getchar(). If only the calls to emit() and the calls to getchar() could be made to feed data to each other, it would be simple to connect the two fragments together so that the output from the decompressor went straight to the parser.

每個程式碼片段都很簡單, 通俗易懂。一個通過呼叫 emit() 方法每次產生一個 byte; 另一個呼叫 getchar() 每次獲取一個 byte 。如果只通過呼叫
emit() 和 getchar() 就能彼此間提交數據, 那麼就能簡單的把兩個程式碼片段連接到一起, 因此也就能把解壓的輸出直接傳遞到解析部分。

 In many modern operating systems, you could do this using pipes between two processes or two threads. emit() in the decompressor writes to a pipe, and getchar() in the parser reads from the other end of the same pipe. Simple and robust, but also heavyweight and not portable. Typically you don't want to have to divide your program into threads for a task this simple.

在很多現在作業系統中, 你可以使用管道在兩個行程或者兩個 thread 間通信。 emit() 在解壓程式碼中寫入管道, 在解析程式碼中使用
getchar() 從另一端相同的管道中讀取。簡單強壯, 但是也重量不可移植。通常你不想把你的一個簡單程序分到 thread 中。

In this article I offer a creative solution to this sort of structure problem. 

在這篇文章中我提供了一個創造性的解決方案來處理這個構造的問題。 

\section{Rewriting}
 The conventional answer is to rewrite one of the ends of the communication channel so that it's a function that can be called. Here's an example of what that might mean for each of the example fragments.

常見的方式是重寫末端的通信管道, 把它作為一個可呼叫的函數。這有個簡單的例子。 

\newpage
\begin{multicols}{2}

\begin{lstlisting}[caption=decompression, basicstyle=\footnotesize]
int decompressor(void) 
{
  static int repchar;
  static int replen;

  if (replen > 0) 
  {
    replen--;
    return repchar;
  }

  c = getchar();

  if (c == EOF)
    return EOF;
  if (c == 0xFF) 
  {
    replen=getchar();
    repchar=getchar();
    replen--;
    return repchar;
  }
  else
    return c;

}
\end{lstlisting}

\begin{lstlisting}[caption=parsesr, basicstyle=\footnotesize, breaklines=true]
void parser(int c) 
{
 static enum 
 {
  START, IN_WORD
 }state;
 switch (state) 
 {
  case IN_WORD:
  if (isalpha(c)) 
  {
   add_to_token(c);
   return;
  }
  got_token(WORD);
  state = START;
  /* fall through */

  case START:
  add_to_token(c);
  if (isalpha(c))
   state = IN_WORD;
  else
   got_token(PUNCT);
  break;
 }
}
\end{lstlisting}
\end{multicols}

 Of course you don't have to rewrite both of them; just one will do. If you rewrite the decompressor in the form shown, so that it returns one character every time it's called, then the original parser code can replace calls to getchar() with calls to decompressor(), and the program will be happy. Conversely, if you rewrite the parser in the form shown, so that it is called once for every input character, then the original decompression code can call parser() instead of emit() with no problems. You would only want to rewrite both functions as callees if you were a glutton for punishment.

當然你不需要兩個都重寫; 只修改一個就可以。如果把解壓片段重寫為以上形式, 那麼每次呼叫它都返回一個字符, 原始的解析程式碼片段中就可以把呼叫 getchar() 改為呼叫 decompressor(), 這樣程序看起來舒心多了。相反的, 如果把解析程式碼重寫為上面的形式, 那麼每次呼叫都會輸入一個字符, 原始的解壓程式碼中可以通過呼叫 parser() 替換掉了呼叫 emit()。如果你是個貪婪的人你可以把兩個函數都重寫, 都作為被呼叫者。

 And that's the point, really. Both these rewritten functions are thoroughly ugly compared to their originals. Both of the processes taking place here are easier to read when written as a caller, not as a callee. Try to deduce the grammar recognised by the parser, or the compressed data format understood by the decompressor, just by reading the code, and you will find that both the originals are clear and both the rewritten forms are less clear. It would be much nicer if we didn't have to turn either piece of code inside out.


這就是我的真實觀點。和兩個重寫後的函數相比原始的程式碼醜陋無比, 在這裡當兩個行程被寫為呼叫者而不是被呼叫者時更易讀了。如果你試圖通過解析器推斷出語法, 或者通過解壓器瞭解解壓數據格式, 只需閱讀下程式碼就可以, 同時你會發現原始的程式碼清晰些, 重寫後的程式碼格式上不太清晰, 如果沒有嵌入另外一者的程式碼這會更好些。 

\section{Knuth's coroutines}
 In The Art of Computer Programming, Donald Knuth presents a solution to this sort of problem. His answer is to throw away the stack concept completely. Stop thinking of one process as the caller and the other as the callee, and start thinking of them as cooperating equals.

在《計算機程序設計藝術》中, Donald Knuth 提供了一個解決這類問題的方法。他的方法是徹底丟掉堆疊的概念, 不要再想一個行程作為呼叫者, 另一個作為被呼叫者, 把他們當做平等的協作者關係。

 In practical terms: replace the traditional "call" primitive with a slightly different one. The new "call" will save the return value somewhere other than on the stack, and will then jump to a location specified in another saved return value. So each time the decompressor emits another character, it saves its program counter and jumps to the last known location within the parser - and each time the parser needs another character, it saves its own program counter and jumps to the location saved by the decompressor. Control shuttles back and forth between the two routines exactly as often as necessary.

實際上就是: 把傳統的「呼叫」稍微改為一個不同的方式。新的「呼叫」將在某個地方保存返回值而不是堆疊上, 並且還能跳轉到另一個保存返回值的指定位置上。因此, 解碼器每次生成一個字符, 就保存它的程序計數器並且跳轉到上次解析器的位置 - 解析器每次都需要一個新的字符, 它保存自己的程序計數器並且跳轉到上次解碼器的位置。程序可以在兩個函數之間來回自如的傳遞需要的數據了。

 This is very nice in theory, but in practice you can only do it in assembly language, because no commonly used high level language supports the coroutine call primitive. Languages like C depend utterly on their stack-based structure, so whenever control passes from any function to any other, one must be the caller and the other must be the callee. So if you want to write portable code, this technique is at least as impractical as the Unix pipe solution.

理論上看起來很美, 但實際中你卻只能在組合語言中使用, 因為通用的高階語言沒有一個支援呼叫原始的協程。像類似於 C 的都是依賴於基礎的堆疊結構, 因此當在函數間進行數據傳遞時, 一個必須作為呼叫者, 另一個必須作為被呼叫者。所以如果你想寫可移植的程式碼, 這種技術和 Unix 管道一樣不切實際。

\section{Stack-based coroutines}

So what we would really like is the ability to mimic Knuth's coroutine call primitive in C. We must accept that in reality, at the C level, one function will be caller and the other will be callee. In the caller, we have no problem; we code the original algorithm, pretty much exactly as written, and whenever it has (or needs) a character it calls the other function.

我們真正想要的是在 C 中模仿 Knuth 協程呼叫原語的能力。我們必須接受這個現實, 在 C 語言中, 一個函數將作為呼叫者, 另一個作為被呼叫者。對於呼叫者我們沒有任何問題;我們按照原始算法寫程式碼就可以, 無論什麼時候, 如果它生成一個字符那麼它就需要呼叫另一個函數。

The callee has all the problems. For our callee, we want a function which has a ``return and continue'' operation: return from the function, and next time it is called, resume control from just after the return statement. For example, we would like to be able to write a function that says


 被呼叫者有很多問題。對於被呼叫者, 我們想要一個函數, 該函數具有「返回並繼續」的操作: 從函數返回後, 當再次呼叫它, 只需要在上次位置繼續執行就可以。比如, 我們希望寫這樣一個函數 

\begin{lstlisting}[basicstyle=\footnotesize, breaklines=true]
int function(void) 
{
  int i;
  for (i = 0; i < 10; i++)
    return i; /* won't work, but wouldn't it be nice */
}
\end{lstlisting}

and have ten successive calls to the function return the numbers 0 through 9.

連續呼叫這個函數 10 次, 返回 0-9 的數字。 

How can we implement this? Well, we can transfer control to an arbitrary point in the function using a goto statement. So if we use a state variable, we could do this:

如何實作這個呢? 好的, 我們可以用一個 goto 敘述句將控制跳轉到這個函數中的任何一點。所以如果我們有一個狀態變數, 我們可以這樣做:

\begin{lstlisting}[basicstyle=\footnotesize, breaklines=true]
int function(void) 
{
  static int i, state = 0;
  switch (state) 
  {
    case 0: goto LABEL0;
    case 1: goto LABEL1;
  }
  LABEL0: /* start of function */
  for (i = 0; i < 10; i++) 
  {
    state = 1; /* so we will come back to LABEL1 */
    return i;
    LABEL1:; /* resume control straight after the return */
  }
}
\end{lstlisting}

This method works. We have a set of labels at the points where we might need to resume control: one at the start, and one just after each return statement. We have a state variable, preserved between calls to the function, which tells us which label we need to resume control at next. Before any return, we update the state variable to point at the right label; after any call, we do a switch on the value of the variable to find out where to jump to.

這個方法可以奏效。在一些我們可能需要恢復控制的地方, 我們擁有一組標籤: 一個在開始位置, 其他的緊跟著每個 return。我們有一個保留在函數呼叫之間的狀態變數, 它指明了下一步我們需要恢復控制那個標籤。在任何返回之前, 我們會更新狀態變數到正確的標籤位;
任何呼叫之後, 我們會對這個變數的值做一個 switch 操作來查看它當前進行到哪裡。

It's still ugly, though. The worst part of it is that the set of labels must be maintained manually, and must be consistent between the function body and the initial switch statement. Every time we add a new return statement, we must invent a new label name and add it to the list in the switch; every time we remove a return statement, we must remove its corresponding label. We've just increased our maintenance workload by a factor of two.


雖然這樣, 它還是比較醜。最糟糕的的部分是, 這一組標籤必須手動維護, 並且在函數體和初始的 switch 敘述句之間保持一致。每次新增一個 return, 我們必須引進一個新表簽名並把它加到 switch 列表中;
而每次刪除一個 return, 我們又必須移除相應的標籤。剛剛我們只是考慮一個因素增加的兩倍維護工作量。

\section{Duff's device}

The famous ``Duff's device'' in C makes use of the fact that a case statement is still legal within a sub-block of its matching switch statement. Tom Duff used this for an optimised output loop: 

C 語言中著名的 ``達夫設備'' 利用 case 敘述句在其匹配 switch 敘述句子塊中也是合法的這一事實。Tom Duff
利用這個方法優化輸出回路:

\begin{lstlisting}[basicstyle=\footnotesize, breaklines=true]
switch (count % 8) 
{
  case 0:  do {  *to = *from++;
  case 7:        *to = *from++;
  case 6:        *to = *from++;
  case 5:        *to = *from++;
  case 4:        *to = *from++;
  case 3:        *to = *from++;
  case 2:        *to = *from++;
  case 1:        *to = *from++;
           } while ((count -= 8) > 0);
}
\end{lstlisting}

We can put it to a slightly different use in the coroutine trick. Instead of using a switch statement to decide which goto statement to execute, we can use the switch statement to perform the jump itself:

我們可以稍加變動將它應用到協同程序技巧上。我們可以用switch敘述句本身執行跳轉, 而不是用它來確定跳到哪裡去執行。 

\begin{lstlisting}[basicstyle=\footnotesize, breaklines=true]
int function(void) 
{
  static int i, state = 0;
  switch (state) 
  {
    case 0: /* start of function */
    for (i = 0; i < 10; i++) 
    {
      state = 1; /* so we will come back to ``case 1'' */
      return i;
      case 1:; /* resume control straight after the return */
    }
  }
}
\end{lstlisting}

Now this is looking promising. All we have to do now is construct a few well chosen macros, and we can hide the gory details in something plausible-looking:

現在這看起來更理想了。我們現在需要做的只是構造一些精確的 macro, 並且可以把細節隱藏到這些似是而非的定義裡: 

\begin{lstlisting}[basicstyle=\footnotesize, breaklines=true]
#define crBegin static int state=0; switch(state) { case 0:
#define crReturn(i,x) do { state=i; return x; case i:; } while (0)
#define crFinish }

int function(void) 
{
  static int i;
  crBegin;
  for (i = 0; i < 10; i++)
    crReturn(1, i);
  crFinish;
}
\end{lstlisting}

(note the use of do ... while(0) to ensure that crReturn does not need braces around it when it comes directly between if and else)

This is almost exactly what we wanted. We can use crReturn to return from the function in such a way that control at the next call resumes just after the return. Of course we must obey some ground rules (surround the function body with crBegin and crFinish; declare all local variables static if they need to be preserved across a crReturn; never put a crReturn within an explicit switch statement); but those do not limit us very much.

這差不多就是我們想要的了。我們可以通過這種一返回就控制下一個呼叫回覆的方式, 用 crReturn
從函數返回。當然, 我們必須遵守一些基本規則 (用 crBegin 和 crFinish 包住函數體; 聲明所需的所有保存在
acrReturn 中的靜態局部變數; 絕不要在一個顯式 switch 敘述句中設置一個 crReturn); 但那些並不會限制我們太多。

The only snag remaining is the first parameter to crReturn. Just as when we invented a new label in the previous section we had to avoid it colliding with existing label names, now we must ensure all our state parameters to crReturn are different. The consequences will be fairly benign - the compiler will catch it and not let it do horrible things at run time - but we still need to avoid doing it.

剩下的唯一問題是傳給 crReturn 的第一個參數。就像在上一節引進一個新標籤一樣, 我們必須避免與已存在的標籤名衝突, 確保所有給 crReturn 的狀態參數都是不同的。這影響是相當小的 -- 編譯器會抓住它並並不讓它在運行時出錯 -- 但我們還是要避免這樣做。

Even this can be solved. ANSI C provides the special macro name \verb+__LINE__+, which expands to the current source line number. So we can rewrite crReturn as

雖然這可以解決, ANSI C 還是提供了擴展到當前行號的專門的 macro:\verb+__LINE__+, 因此我們可以把 crReturn 重寫成:

\begin{lstlisting}[basicstyle=\footnotesize, breaklines=true]
#define crReturn(x) do { state=__LINE__; return x; \
                         case __LINE__:; } while (0)
\end{lstlisting}

and then we no longer have to worry about those state parameters at all, provided we obey a fourth ground rule (never put two crReturn statements on the same line).

而且只要遵守第四條基本規則, 我們就不再需要擔心這些狀態參數了 (決不在同一行寫兩個 crReturn 敘述句)。 

\section{Evaluation}

So now we have this monstrosity, let's rewrite our original code fragments using it.

現在我們有了這個畸形的東西, 讓我們來重寫原始的程式碼片段吧。 

\newpage
\begin{multicols}{2}
\begin{lstlisting}[caption=decompressor, basicstyle=\footnotesize, breaklines=true]
int decompressor(void) 
{
  static int c, len;
  crBegin;

  while (1) 
  {
    c = getchar();

    if (c == EOF)
      break;
    if (c == 0xFF) 
    {
      len = getchar();
      c = getchar();
      while (len--)
        crReturn(c);
    } 
    else
      crReturn(c);
  }
  crReturn(EOF);
  crFinish;
}
\end{lstlisting}


\begin{lstlisting}[caption=parser, basicstyle=\footnotesize, breaklines=true]
void parser(int c) 
{
  crBegin;
  while (1) 
  {
    /* first char already in c */
    if (c == EOF)
      break;
    if (isalpha(c)) 
    {
      do 
      {
        add_to_token(c);
        crReturn( );
      } while (isalpha(c));
      got_token(WORD);
    }
    add_to_token(c);
    got_token(PUNCT);
    crReturn( );
  }
  crFinish;
}
\end{lstlisting}
\end{multicols}

We have rewritten both decompressor and parser as callees, with no need at all for the massive restructuring we had to do last time we did this. The structure of each function exactly mirrors the structure of its original form. A reader can deduce the grammar recognised by the parser, or the compressed data format used by the decompressor, far more easily than by reading the obscure state-machine code. The control flow is intuitive once you have wrapped your mind around the new format: when the decompressor has a character, it passes it back to the caller with crReturn and waits to be called again when another character is required. When the parser needs another character, it returns using crReturn, and waits to be called again with the new character in the parameter c.

我們已經重寫瞭解碼器和解析器都作為了被呼叫者, 不需要像最後一次那樣大規模的重寫。每個函數的結構恰好是原始結構的鏡像, 讀者能夠通過解析器推斷出相應的語法, 或者通過解碼器瞭解到解壓數據格式, 比起讀那些狀態機程式碼簡單多了。一旦你適應了新的格式, 你就會發現控制流程非常直觀:當解壓器產生一個字符, 它就呼叫 crReturn 將這個字符傳給呼叫函數, 並等待當需要下一個字符時再次被呼叫。當解析器需要一個新字符時, 它通過
crReturn 返回, 並等待下一次被呼叫時通過參數 c 傳入新的字符。

There has been one small structural alteration to the code: parser() now has its getchar() (well, the corresponding crReturn) at the end of the loop instead of the start, because the first character is already in c when the function is entered. We could accept this small change in structure, or if we really felt strongly about it we could specify that parser() required an ``initialisation'' call before you could start feeding it characters.

這裡程式碼裡還有一個小的結構變化: parser() 中 getchar() 放在了迴圈的結尾而不是開頭了, 這是因為當進入函數時, 第一個字符已經通過 c
傳進來了。我們應該可以接受這個小的結構改變, 或者如果我們真的對此抱有輕盈態度, 我們可以認為在 parse()
傳入數據前, 需要一次「初始化」呼叫。 

As before, of course, we don't have to rewrite both routines using the coroutine macros. One will suffice; the other can be its caller.

當然像前面一樣, 我們不必把兩個函數都使用協程的 macro 重寫, 修改一個足矣; 另一個可以作為他的呼叫者。

We have achieved what we set out to achieve: a portable ANSI C means of passing data between a producer and a consumer without the need to rewrite one as an explicit state machine. We have done this by combining the C preprocessor with a little-used feature of the switch statement to create an implicit state machine.


我們已經取得了我們想要達到的目標: 一個可移植的 ANSI C 在生產者和消費者之間傳遞數據, 而不用狀態機重寫程式碼。我們已經通過把一個
switch 敘述句的生僻的功能結合 C 的預處理創建一個隱式的狀態機。 

\section{Coding Standards}

Of course, this trick violates every coding standard in the book. Try doing this in your company's code and you will probably be subject to a stern telling off if not disciplinary action! You have embedded unmatched braces in macros, used case within sub-blocks, and as for the crReturn macro with its terrifyingly disruptive contents . . . It's a wonder you haven't been fired on the spot for such irresponsible coding practice. You should be ashamed of yourself.

當然, 這種方式違背了書上的編碼規範, 在公司程式碼中嘗試這樣使用你會受到嚴厲的警告! 在 macro 定義中, 你的大括號沒有完全匹配, 子區塊中使用了 case, 至於 crReturn 宏中內容不完整 ... 使用這種不負責任的編碼實踐你沒有被解僱真是一種奇蹟。你應該感到自行慚愧。 

I would claim that the coding standards are at fault here. The examples I've shown in this article are not very long, not very complicated, and still just about comprehensible when rewritten as state machines. But as the functions get longer, the degree of rewriting required becomes greater and the loss of clarity becomes much, much worse.

我需要聲明下編碼規範在這裡不適用。在這篇文章裡我展示的例子不是很長, 也不複雜, 即使使用狀態機重寫也能看懂。但是隨著函數變長, 重寫的難度越來愈大, 並且失去了程式碼清晰度, 越來越糟糕。 

Consider. A function built of small blocks of the form

考慮下, 如果一個函數有下面的程式碼塊構成 

\begin{lstlisting}[basicstyle=\footnotesize, breaklines=true]
case STATE1:
/* perform some activity */
if (condition) state = STATE2; else state = STATE3;
\end{lstlisting}

is not very different, to a reader, from a function built of small blocks of the form

 對於讀者來說從函數建立下面的小模塊也不是很難 

\begin{lstlisting}[basicstyle=\footnotesize, breaklines=true]
LABEL1:
/* perform some activity */
if (condition) goto LABEL2; else goto LABEL3;
\end{lstlisting}

One is caller and the other is callee, true, but the visual structure of the functions are the same, and the insights they provide into their underlying algorithms are exactly as small as each other. The same people who would fire you for using my coroutine macros would fire you just as loudly for building a function out of small blocks connected by goto statements! And this time they would be right, because laying out a function like that obscures the structure of the algorithm horribly.

 一個呼叫者, 一個被呼叫者。是的, 這兩個函數在可視結構上是一樣的, 它們提供的底層算法都非常少。因為使用 coroutine macro 想要解僱你的人同樣會因為你寫的連接 goto 敘述句的小塊函數而怒斥你! 這次它們做對了, 因為這樣的函數佈局破壞了算法的結構。 

Coding standards aim for clarity. By hiding vital things like switch, return and case statements inside ``obfuscating'' macros, the coding standards would claim you have obscured the syntactic structure of the program, and violated the requirement for clarity. But you have done so in the cause of revealing the algorithmic structure of the program, which is far more likely to be what the reader wants to know!

編程規範目標就是為了清晰。像 switch, return 和 case 敘述句把這些重要的東西隱藏到「不清晰」的宏中, 從編程標準角度來看你已經破壞了程序的語法結構, 違背了程式碼清晰的要求。但是我們這樣做是為了突出程序的算法結構, 而算法結構也正好是讀者想要瞭解的!

Any coding standard which insists on syntactic clarity at the expense of algorithmic clarity should be rewritten. If your employer fires you for using this trick, tell them that repeatedly as the security staff drag you out of the building.

 任何的編碼規範堅持語法的清晰度而犧牲了算法的清晰度都應該被重寫。如果你的老闆因為使用這個技巧而解僱你, 當安保人員把你拖走時要不斷告訴他們。 
\section{Refinements and Code}

In a serious application, this toy coroutine implementation is unlikely to be useful, because it relies on static variables and so it fails to be re-entrant or multi-threadable. Ideally, in a real application, you would want to be able to call the same function in several different contexts, and at each call in a given context, have control resume just after the last return in the same context.

 在正兒八經的應用程序中, 協程很少使用, 因為它依賴於靜態變數, 因此不能重入或者支持多 thread。理想情況下, 在真實應用程序中, 你可能想在多個不同的上下文中呼叫同一個函數, 並且在給定的一個上下文中每次呼叫時, 都能在這個上下文中上一次返回的位置繼續執行。 

This is easily enough done. We arrange an extra function parameter, which is a pointer to a context structure; we declare all our local state, and our coroutine state variable, as elements of that structure.

 這很容易做到, 我們增加一個新的函數參數——一個上下文結構的指針;我們將所有的局部變數和協程用到的狀態變數都聲明成結構體中的元素. 

It's a little bit ugly, because suddenly you have to use \verb+ctx->i+ as a loop counter where you would previously just have used i; virtually all your serious variables become elements of the coroutine context structure. But it removes the problems with re-entrancy, and still hasn't impacted the structure of the routine.

這樣寫醜了點, 因為突然間你不得不使用 \verb+ctx->i+ 作為循環計數器而不是之前使用的 i;
事實上所有重要的變數都成了協程上下文結構體中的元素。但是這消除了重入的問題, 並且沒有影響到程序的結構。 

(Of course, if C only had Pascal's with statement, we could arrange for the macros to make this layer of indirection truly transparent as well. A pity. Still, at least C++ users can manage this by having their coroutine be a class member, and keeping all its local variables in the class so that the scoping is implicit.)

(當然, 要是 C 有 Pascal 語言的 with 敘述句, 我們就可以將這個間接的引用隱藏掉, 但是很遺憾沒有這些。對於 C++
語言, 我們可以把協程的兩個函數設計成類的成員函數, 所有的局部變數設計成類的成員變數, 從而將作用域的問題隱藏掉) 

Included here is a C header file that implements this coroutine trick as a set of pre-defined macros. There are two sets of macros defined in the file, prefixed scr and ccr. The scr macros are the simple form of the technique, for when you can get away with using static variables; the ccr macros provide the advanced re-entrant form. Full documentation is given in a comment in the header file itself.

這兒包含的 C 語言頭文件實現了一套預定義的協程使用的宏。在文件中定義了二套宏函數, 前綴分別是 scr 和 ccr。scr macro 是一套簡單的實現, 用於可以使用靜態變數的情況; ccr macro 更高級一些, 能支援重入。在頭文件的註釋中有完整的說明。 

Note that Visual C++ version 6 doesn't like this coroutine trick, because its default debug state (Program Database for Edit and Continue) does something strange to the
\verb+__LINE__+ macro. To compile a coroutine-using program with VC++ 6, you must turn off Edit and Continue. (In the project settings, go to the ``C/C++'' tab, category ``General'', setting ``Debug info''. Select any option other than ``Program Database for Edit and Continue''.)

需要注意的是, VC++ 6 並不喜歡這種協程技巧, 因為其預設的 debug 狀態(Program Database for Edit and Continue)對 \verb+__LINE__+ macro
的支持有點兒怪。如果想用 VC++ 6 編譯一個使用了協程的 macro, 你就要關掉 Edit and Continue。(在 project settings 中, 選擇 c/c++ 標籤,
在 General 中, 將Debug info 改為 Program Database for Edit and Continue 之外的其他值)。 

(The header file is MIT-licensed, so you can use it in anything you like without restriction. If you do find something the MIT licence doesn't permit you to do, mail me, and I'll probably give you explicit permission to do it anyway.)

(這個頭文件是 MIT 授權, 所以你可以任意使用, 沒有任何限制。如果你發現 MIT 對你的使用方式有限制, 可以給我發郵件, 我會考慮給你授權。)

Follow this link for coroutine.h (page \pageref{ref_coroutine_h}, \ref{ref_coroutine_h}).


Thanks for reading. Share and enjoy!

\section{References}

\begin{itemize}
\item Donald Knuth, The Art of Computer Programming, Volume 1. Addison-Wesley, ISBN 0-201-89683-4. Section 1.4.2 describes coroutines in the ``pure'' form.

\item http://www.lysator.liu.se/c/duffs-device.html is Tom Duff's own discussion of Duff's device. Note, right at the bottom, a hint that Duff might also have independently invented this coroutine trick or something very like it.


Update, 2005-03-07: Tom Duff confirms this in a blog comment. The ``revolting way to use switches to implement interrupt driven state machines'' of which he speaks in his original email is indeed the same trick as I describe here.

\item PuTTY is a Win32 Telnet and SSH client. The SSH protocol code contains real-life use of this coroutine trick. As far as I know, this is the worst piece of C hackery ever seen in serious production code. 

\end{itemize}


\begin{itemize}
\item    Donald Knuth,計算機編程藝術, 卷 1. Addison-Wesley, ISBN 0-201-89683-4. Section 1.4.2 describes coroutines in the "pure" form.
\item        http://www.lysator.liu.se/c/duffs-device.html 是 Tom Duff's 自己關於 Duff 策略的討論。注意, 右下角, 暗示著 Duff 有可能獨立的完成這樣的協程技巧或者其他類似的東西。

2005-03-07 更新: Tom Duff 在一篇博客評論中證實了這一點。「revolting way to use switches to implement interrupt driven state machines」這篇文章中和他最初在郵件中說的確實一樣.
\item                PuTTY 是一個 Win32 Telnet 和 SSH 客戶端。SSH 協議其中一部分使用了協程技巧。據我所知, 這是一個在生產程式碼中最糟糕的
C 程式碼片段的輪子。
\end{itemize}




\appendix
\appendixpage
\section{coroutine.h}\label{ref_coroutine_h}

%\lstinputlisting[basicstyle=\footnotesize,frame=single,frameround=tttt,numbers=left,breaklines=true,title=coroutine.h]{coroutine.h}
\lstinputlisting[basicstyle=\footnotesize,numbers=left,breaklines=true,title=coroutine.h]{coroutine.h}
%\lstinputlisting[basicstyle=\footnotesize,frame=single,frameround=tttt,breaklines=true,title=modify a little version]{coroutine.h}

\end{CJK}
\end{document}
