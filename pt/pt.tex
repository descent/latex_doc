\documentclass[12pt]{article}
%\usepackage[paperwidth=8,paperheight=11cm,textwidth=13cm,textheight=14cm,left=0.5cm,right=0.5cm,top=0.5cm,nohead,nofoot]{geometry} 


\usepackage[paperwidth=10.1cm,paperheight=13cm,left=0.3cm,right=0.3cm,bottom=0.5cm,top=0.1cm,nohead,nofoot]{geometry} 

%\usepackage[papersize={4.5in,6in},margin=0.5cm]{geometry}
%\usepackage[body={2cm, 5.5in},left=0.2in,right=0.2in,top=0.2in,bottom=0.2in]{geometry}
\usepackage{multicol}

\parindent=0cm
\parskip=13pt
% 轉成 PDF 時, 產生 hyperlink 的效果
%\usepackage{CJK}
\usepackage{CJKutf8} %使用 CJKutf8 套件, 配合, \hypersetup 產生中文 pdf 書籤
%\usepackage[dvips]{hyperref}
\usepackage{type1cm}
%\usepackage{fancyvrb}
\usepackage{graphicx}
\usepackage{lastpage}
\usepackage[dvips]{color}
\usepackage{listings}
\usepackage{comment}
%\usepackage{ucs}
%\usepackage[utf8]{inputenc}
% 在使用 latex2html 可以處理中文
%\begin{htmlonly}
%\usepackage{taiwan}
%\end{htmlonly}
%\usepackage{cwtex}
%\begin{htmlonly}
%\usepackage{html}
%\end{htmlonly}
\usepackage{appendix}

%\usepackage[dvips,CJKbookmarks,pdftitle={From TeX/LaTeX to PDF}%
%                ,pdfsubject={From TeX/LaTeX to PDF}%
%		 ,pdfkeywords={tex,latex,cjk,typeset,pdf,typography}]{hyperref}
\usepackage{hyperref} % 最好保证 hyperref 是最后加载的宏包
\hypersetup{%
  dvipdfmx,% 设定要使用的 driver 为 dvipdfmx
  unicode={true},% 使用 unicode 来编码 PDF 字符串
  pdfstartview={FitH},% 文档初始视图为匹配宽度
  bookmarksnumbered={true},% 书签附上章节编号
  bookmarksopen={true},% 展开书签
  pdfborder={0 0 0},% 链接无框
  pdftitle={Protothreads}
  pdfauthor={Adam Dunkels}
  pdfsubject={Protothreads}
  pdfkeywords={Protothreads},
  pdfcreator={应用程序},
  pdfproducer={PDF 制作程序},% 这个好像没起作用?
}

%\begin{htmlonly}

%\renewcommand{\abstractname}{\r18 摘要}
%\renewcommand{\figurename}{\m11 圖}
%\renewcommand{\contentsname}{\r20 目錄}

%\end{htmlonly}

%\renewcommand{\appendixpagenam}{附錄}

% 標題頁
% 正文
\title{Protothreads}
\author{Adam Dunkels <adam@sics.se>}
\begin{document}
\maketitle
%\begin{CJK}{UTF8}{cwku}
%\renewcommand{\contentsname}{目錄}

%\begin{htmlonly}
%\include{auth}
%\end{htmlonly}

\newpage
%\addtocontents{toc}{text}
\tableofcontents
%\addcontentsline{toc}{section}{abstract}
%\addcontentsline{toc}{section}{REFERENCE}
\newpage

\begin{CJK}{UTF8}{bsmi} %開始CJK環境, 設定編碼, 設定字體

\section{The Protothreads Library}

Author:

Adam Dunkels \verb+<adam@sics.se>+

Protothreads are a type of lightweight stackless threads designed for severly memory constrained systems such as deeply embedded systems or sensor network nodes. Protothreads provides linear code execution for event-driven systems implemented in C. Protothreads can be used with or without an RTOS.

Protothreads are a extremely lightweight, stackless type of threads that provides a blocking context on top of an event-driven system, without the overhead of per-thread stacks. The purpose of protothreads is to implement sequential flow of control without complex state machines or full multi-threading. Protothreads provides conditional blocking inside C functions.

Main features:

\begin{itemize}
\item  No machine specific code - the protothreads library is pure C
\item  Does not use error-prone functions such as 

longjmp()
\item  Very small RAM overhead - only two bytes per protothread
\item  Can be used with or without an OS
\item  Provides blocking wait without full 

multi-threading or stack-switching
\end{itemize}


Examples applications:

\begin{itemize}
\item Memory constrained systems
\item Event-driven protocol stacks
\item Deeply embedded systems
\item Sensor network nodes
\end{itemize}

See also:
Example programs
Protothreads API documentation

The protothreads library is released under a BSD-style license that allows for both non-commercial and commercial usage. The only requirement is that credit is given.

More information and new version of the code can be found at the Protothreads homepage:

http://www.sics.se/∼adam/pt/

\subsection{Authors}

The protothreads library was written by Adam Dunkels \verb+<adam@sics.se>+ with support from Oliver Schmidt \verb+<ol.sc@web.de>+.

\subsection{Using protothreads}

Using protothreads in a project is easy: simply copy the files {\bf pt.h}, {\bf lc.h} and {\bf lc-switch.h} into the include files directory of the project, and \verb+#include "pt.h"+ in all files that should use protothreads.

\subsection{Protothreads}

Protothreads are a extremely lightweight, stackless threads that provides a blocking context on top of an event-driven system, without the overhead of per-thread stacks. The purpose of protothreads is to implement sequential flow of control without using complex state machines or full multi-threading. Protothreads provides conditional blocking inside a C function.

In memory constrained systems, such as deeply embedded systems, traditional multi-threading may have a too large memory overhead. In traditional multi-threading, each thread requires its own stack, that typically is over-provisioned. The stacks may use large parts of the available memory.

The main advantage of protothreads over ordinary threads is that protothreads are very lightweight: a protothread does not require its own stack. Rather, all protothreads run on the same stack and context switching is done by stack rewinding. This is advantageous in memory constrained systems, where a stack for a thread might use a large part of the available memory. A protothread only requires only two bytes of memory per protothread. Moreover, protothreads are implemented in pure C and do not require any machine-specific assembler code.

A protothread runs within a single C function and cannot span over other functions. A protothread may call normal C functions, but cannot block inside a called function. Blocking inside nested function calls is instead made by spawning a separate protothread for each potentially blocking function. The advantage of this approach is that blocking is explicit: the programmer knows exactly which functions that block that which functions the never blocks.

Protothreads are similar to asymmetric co-routines. The main difference is that co-routines uses a separate stack for each co-routine, whereas protothreads are stackless. The most similar mechanism to protothreads are Python generators. These are also stackless constructs, but have a different purpose. Protothreads provides blocking contexts inside a C function, whereas Python generators provide multiple exit points from a generator function.


\subsection{Local variables}

Note:
\begin{quote}
Because protothreads do not save the stack context across a blocking call, local variables are not preserved when the protothread blocks. This means that local variables should be used with utmost care - if in doubt, do not use local variables inside a protothread!
\end{quote}

\subsection{Scheduling}

A protothread is driven by repeated calls to the function in which the protothread is running. Each time the function is called, the protothread will run until it blocks or exits. Thus the scheduling of protothreads is done by the application that uses protothreads.

\subsection{Implementation}

Protothreads are implemented using local continuations. A local continuation represents the current state of execution at a particular place in the program, but does not provide any call history or local variables.  A local continuation can be set in a specific function to capture the state of the function. After a local continuation has been set can be resumed in order to restore the state of the function at the point where the local continuation was set.

Local continuations can be implemented in a variety of ways:

\begin{enumerate}
\item by using machine specific assembler code,
\item by using standard C constructs, or
\item by using compiler extensions.
\end{enumerate}

The first way works by saving and restoring the processor state, except for stack pointers, and requires between 16 and 32 bytes of memory per protothread. The exact amount of memory required depends on the architecture.

The standard C implementation requires only two bytes of state per protothread and utilizes the C switch() statement in a non-obvious way that is similar to Duff’s device. This implementation does, however, impose a slight restriction to the code that uses protothreads: a protothread cannot perform a blocking wait (\verb+PT_WAIT_UNTIL()+ or \verb+PT_YIELD()+) inside a switch() statement.

Certain compilers has C extensions that can be used to implement protothreads. GCC supports label pointers that can be used for this purpose. With this implementation, protothreads require 4 bytes of RAM per protothread.

\section{The Protothreads Library 1.4 Module Index}

\subsection{The Protothreads Library 1.4 Modules}

Here is a list of all modules:

\begin{itemize}
\item Protothreads 
\item Protothread semaphores 
\item Local continuations
\item Examples 
\end{itemize}

\section{The Protothreads Library 1.4 Hierarchical Index}

\subsection{The Protothreads Library 1.4 Class Hierarchy}

This inheritance list is sorted roughly, but not completely, alphabetically:

pt \\
\verb+pt_sem+

\section{The Protothreads Library 1.4 Data Structure Index}
\subsection{The Protothreads Library 1.4 Data Structures}

Here are the data structures with brief descriptions:

pt\\
\verb+pt_sem+

\section{The Protothreads Library 1.4 File Index}

\subsection{The Protothreads Library 1.4 File List}

Here is a list of all documented files with brief descriptions:

\begin{description} 
\item [lc-addrlabels.h] (Implementation of local continuations based on the "Labels as values" feature of gcc ) 

\item [lc-switch.h] (Implementation of local continuations based on switch() statment ) 26

\item[lc.h] (Local continuations ) 

\item [pt-sem.h] (Couting semaphores implemented on protothreads )
\item [pt.h] (Protothreads implementation )
\end{description} 

\section{The Protothreads Library 1.4 Module Documentation}

\subsection{Protothreads}

\subsubsection{Detailed Description}

Protothreads are implemented in a single header file, pt.h, which includes the local continuations header file, lc.h.

This file in turn includes the actual implementation of local continuations, which typically also is contained in a single header file.

\bf Files

\begin{itemize}
\item file pt.h 

Protothreads implementation.
\end{itemize}


\bf Modules

\begin{itemize}
\item Protothread semaphores 

This module implements counting semaphores on top of protothreads.

\item Local continuations 

Local continuations form the basis for implementing protothreads.
\end{itemize}

\bf Data Structures

\begin{itemize}
\item struct pt
\end{itemize}

\bf Initialization

\begin{itemize}
\item \verb+#define PT_INIT(pt)+

Initialize a protothread.
\end{itemize}


\bf Declaration and definition

\begin{itemize}
\item \verb+#define PT_THREAD(name_args)+

Declaration of a protothread.

\item \verb+#define PT_BEGIN(pt)+

Declare the start of a protothread inside the C function implementing the protothread.

\item \verb+#define PT_END(pt)+

Declare the end of a protothread.

\end{itemize}


\bf Blocked wait

\begin{itemize}
\item \verb+#define PT_WAIT_UNTIL(pt, condition)+

Block and wait until condition is true.

\item \verb+#define PT_WAIT_WHILE(pt, cond)+

Block and wait while condition is true.
\end{itemize}

\bf Hierarchical protothreads

\begin{itemize}
\item \verb+#define PT_WAIT_THREAD(pt, thread)+

Block and wait until a child protothread completes.

\item \verb+#define PT_SPAWN(pt, child, thread)+

Spawn a child protothread and wait until it exits.
\end{itemize}

\bf Exiting and restarting

\begin{itemize}
\item \verb+#define PT_RESTART(pt)+

Restart the protothread.

\item \verb+#define PT_EXIT(pt)+

Exit the protothread.
\end{itemize}

\bf Calling a protothread

\begin{itemize}
\item \verb+#define PT_SCHEDULE(f)+

Schedule a protothread.
\end{itemize}

\bf Yielding from a protothread

\begin{itemize}
\item \verb+#define PT_YIELD(pt)+

Yield from the current protothread.

\item \verb+#define PT_YIELD_UNTIL(pt, cond)+

Yield from the protothread until a condition occurs.
\end{itemize}

\bf Defines

\begin{itemize}
\item \verb+#define PT_WAITING 0+
\item \verb+#define PT_YIELDED 1+
\item \verb+#define PT_EXITED 2+
\item \verb+#define PT_ENDED 3+
\end{itemize}


\subsubsection{Define Documentation}

\subsubsection{\#define PT\_{}BEGIN(pt)}
Declare the start of a protothread inside the C function implementing the protothread.

This macro is used to declare the starting point of a protothread. It should be placed at the start of the function in which the protothread runs. All C statements above the \verb+PT_BEGIN()+ invokation will be executed each time the protothread is scheduled.


Parameters:

pt A pointer to the protothread control structure.

Definition at line 115 of file pt.h.

\subsubsection{\#define PT\_{}END(pt)}

Declare the end of a protothread.

This macro is used for declaring that a protothread ends. It must always be used together with a matching \verb+PT_BEGIN()+ macro.

Parameters:

pt A pointer to the protothread control structure.

Definition at line 127 of file pt.h.

\subsubsection{\#define PT\_EXIT(pt)}

Exit the protothread.

This macro causes the protothread to exit. If the protothread was spawned by another protothread, the

parent protothread will become unblocked and can continue to run.

Parameters:

pt A pointer to the protothread control structure.

Definition at line 246 of file pt.h.

\subsubsection{\#define PT\_INIT(pt)}

Initialize a protothread.

Initializes a protothread. Initialization must be done prior to starting to execute the protothread.

Parameters:

pt A pointer to the protothread control structure.

See also:

\verb+PT_SPAWN()+

Definition at line 80 of file pt.h.

\subsubsection{\#define PT\_RESTART(pt)}

Restart the protothread.

This macro will block and cause the running protothread to restart its execution at the place of the \verb+PT_-BEGIN()+ call.

Parameters:

pt A pointer to the protothread control structure.

Definition at line 229 of file pt.h.

Generated on Mon Oct 2 10:06:29 2006 for The Protothreads Library 1.4 by Doxygen6.1 Protothreads 8

\subsubsection{\#define PT\_SCHEDULE(f)}

Schedule a protothread.

This function shedules a protothread. The return value of the function is non-zero if the protothread is

running or zero if the protothread has exited.

Parameters:

f The call to the C function implementing the protothread to be scheduled

Definition at line 271 of file pt.h.

\subsubsection{\#define PT\_SPAWN(pt, child, thread)}

Spawn a child protothread and wait until it exits.

This macro spawns a child protothread and waits until it exits. The macro can only be used within a

protothread.

Parameters:

pt A pointer to the protothread control structure.

child A pointer to the child protothread’s control structure.

thread The child protothread with arguments

Definition at line 206 of file pt.h.

\subsubsection{\#define PT\_THREAD(name\_args)}

Declaration of a protothread.

This macro is used to declare a protothread. All protothreads must be declared with this macro.

Parameters:

\verb+name_args+ The name and arguments of the C function implementing the protothread.

Definition at line 100 of file pt.h.

\subsubsection{\#define PT\_WAIT\_THREAD(pt, thread)}

Block and wait until a child protothread completes.

This macro schedules a child protothread. The current protothread will block until the child protothread completes.

Note:

The child protothread must be manually initialized with the \verb+PT_INIT()+ function before this function is used.

Parameters:

pt A pointer to the protothread control structure.

thread The child protothread with arguments

See also:

\verb+PT_SPAWN()+

Definition at line 192 of file pt.h.

\subsubsection{\#define PT\_WAIT\_UNTIL(pt, condition)}

Block and wait until condition is true.

This macro blocks the protothread until the specified condition is true.

Parameters:

pt A pointer to the protothread control structure.

condition The condition.

Definition at line 148 of file pt.h.


\subsubsection{\#define PT\_WAIT\_WHILE(pt, cond)}

Block and wait while condition is true.

This function blocks and waits while condition is true. See \verb+PT_WAIT_UNTIL()+.

Parameters:

pt A pointer to the protothread control structure.

cond The condition.

Definition at line 167 of file pt.h.

\subsubsection{\#define PT\_YIELD(pt)}

Yield from the current protothread.

This function will yield the protothread, thereby allowing other processing to take place in the system.

Parameters:

pt A pointer to the protothread control structure.

Definition at line 290 of file pt.h.

\subsubsection{\#define PT\_YIELD\_UNTIL(pt, cond)}

Yield from the protothread until a condition occurs.

Parameters:

pt A pointer to the protothread control structure.

cond The condition.

This function will yield the protothread, until the specified condition evaluates to true.

Definition at line 310 of file pt.h.

\subsection{Examples}

\subsubsection{A small example}

This first example shows a very simple program: two protothreads waiting for each other to toggle two flags. The code illustrates how to write protothreads code, how to initialize protothreads, and how to schedule them.



\end{CJK}
\end{document}
