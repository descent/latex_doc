% test for cwtex font is ok.

\documentclass[12pt]{article}
%\usepackage{CJK}
\usepackage{CJKutf8}
\usepackage{comment}
\usepackage{ucs}
\usepackage[utf8]{inputenc}
\usepackage{hyperref} % 最好保證 hyperref 是最後載入的 macro
\hypersetup{%
%  dvipdfmx,% 设定要使用的 driver 为 dvipdfmx
  dvips,% 设定要使用的 driver 为 dvipdfmx
  unicode={true},% 使用 unicode 来编码 PDF 字符串
  pdfstartview={FitH},% 文档初始视图为匹配宽度
  bookmarksnumbered={true},% 书签附上章节编号
  bookmarksopen={true},% 展开书签
  pdfborder={0 0 0},% 链接无框
  pdftitle={日文學習心得},
  pdfauthor={descent},
  pdfsubject={日文學習},
  pdfkeywords={日文},
  pdfcreator={ps2pdf},
  pdfproducer={PDF 制作程序},% 这个好像没起作用?
}

%\usepackage{html}

\begin{document}
%\begin{CJK}{UTF8}{cwku}
\begin{CJK}{UTF8}{nsung}
\renewcommand{\contentsname}{目錄}
\renewcommand{\tablename}{表}
\renewcommand{\figurename}{圖}
\renewcommand{\listtablename}{表格目錄}
\renewcommand{\listfigurename}{圖目錄}

\tableofcontents
\newpage


\section{五十音}

\begin{table}[htdp]
%begin{latexonly}
\caption{平假名}
%end{latexonly}
\begin{tabular}{cccccccccc}
\hline
あ& か & さ & た & な & は & ま & や & ら & わ\\
a & ka & sa & ta & na & ha & ma& ya & ra & wa\\
\hline
い & き & し & ち & に & ひ & み & & り& \\
i & ki & shi & chi & ni & hi & mi & & ri& \\
\hline
う & く  & す & つ & ぬ & ふ & む & ゆ & る & ん\\	
 u & ku &  su &  tsu &  nu &  fu &  mu &  yu &  ru & n\\	
\hline
え & け & せ & て & ね & へ & め & & る & \\	
 r & ke &  se &  te &  ne &  he &  me & &  ru & \\	
\hline
お & こ  & そ & と & の & ほ & も & よ & ろ & を\\	
 o &  ko &  so &  to &  no &  ho &  mo &  yo &  ro & o\\	

\hline
\end{tabular}
\end{table}


\begin{table}[htdp]
%begin{latexonly}
\caption{片假名}
%end{latexonly}
\begin{tabular}{cccccccccc}
\hline
ア& カ & サ & タ & ナ & ハ & マ & ヤ & ラ & ワ\\
a & ka & sa & ta & na & ha & ma& ya & ra & wa\\
\hline
イ& キ & シ & チ & ニ & ヒ & ミ &  & リ & \\
i & ki & shi & chi & ni & hi & mi & & ri& \\
\hline
ウ& ク & ス & ス & ヌ & フ & ム & ユ & ル & ン\\
 u & ku &  su &  tsu &  nu &  fu &  mu &  yu &  ru & n\\	
\hline
エ& ケ & セ & テ & ネ & ヘ & メ & & レ & \\
 r & ke &  se &  te &  ne &  he &  me & &  ru & \\	
\hline
オ& コ & ソ & ト & ノ & ホ & モ & ヨ & ロ & ヲ\\
 o &  ko &  so &  to &  no &  ho &  mo &  yo &  ro & o\\	

\hline
\end{tabular}
\end{table}

\section{動詞變化}

\subsection{五段動詞語尾變化}
\begin{table}[htdp]
\begin{tabular}{cccccc}
第一變化 & 第二變化 & 第三變化 & 第四變化 & 第五變化 & 第六變化 \\
\hline
ア/オ 段音 & イ 段音 & ウ 段音 & ウ 段音  & エ 段音 & エ 段音  \\
\end{tabular}
\end{table}

\subsection{上一段動詞語尾變化}
\begin{table}[htdp]
\begin{tabular}{cccccc}
第一變化 & 第二變化 & 第三變化 & 第四變化 & 第五變化 & 第六變化 \\
\hline
イ 段音 & イ 段音 & イ 段音 + る &  イ 段音 + る &  イ 段音 + れ &  段音 + ろ/よ \\
\end{tabular}
\end{table}

\subsection{か行變格動詞語尾變化}
\begin{table}[htdp]
\begin{tabular}{cccccc}
第一變化 & 第二變化 & 第三變化 & 第四變化 & 第五變化 & 第六變化 \\
\hline
こ & き & くる & くる & くれ& こい \\
来 & 来 & 来る & 来る & 来れ& 来い \\
\end{tabular}
\end{table}

\begin{comment}
\end{comment}
\end{CJK}

\end{document}
