% http://daiyuwen.freeshell.org/gb/rol/roots_of_lisp.tex.gz
%\documentclass[12pt, a4]{article}
\documentclass[12pt]{article}
\usepackage[paperwidth=10.1cm, paperheight=13cm, left=0.3cm, right=0.3cm, bottom=0.5cm, top=0.1cm, nohead, nofoot]{geometry}


\usepackage{CJKutf8} %使用CJK套件
\usepackage{comment}


\begin{document} 
\begin{CJK}{UTF8}{bsmi} %開始CJK環境,設定編碼,設定字體
%\CJKtilde 
%\CJKindent 
%\CJKcaption{GB} 
\title{Lisp 之根源 (The Roots of Lisp)} 
\author{保羅格雷厄姆 (Paul Graham)} 
\maketitle 
%\begin{comment}
% some abbreviations 
\newcommand{\pone}{$p_{1}$} 
\newcommand{\pn}{$p_{n}$} 
\newcommand{\aone}{$a_{1}$} 
\newcommand{\an}{$a_{n}$} 
\newcommand{\vone}{$v_{1}$} 
\newcommand{\vn}{$v_{n}$} 
\newcommand{\eone}{$e_{1}$} 
\newcommand{\en}{$e_{n}$} 
約翰麥卡錫於1960年發表了一篇非凡的論文, 他在這篇論文中對編程的貢獻有如 
歐幾里德對幾何的貢獻.\footnote{``Recursive Functions of Symbolic Expressions and Their Computation by Machine, Part1.'' 
{\it Communication of the ACM} 3:4, April 1960, pp. 184-195.} 
他向我們展示了, 在只給定幾個簡單的操作符和一個 
表示函數的記號的基礎上, 
如何構造出一個完整的編程語言. 
麥卡錫稱這種語 
言為Lisp, 
意為 
List 
Processing, 
因為他的主要思想之一是用一種簡單的數據 
結構表(list)來代表代碼和數據. 

值得注意的是, 麥卡錫所作的發現, 不僅是計算機史上劃時代的大事, 
而且是一種在我們這個時代編程越來越趨向的模式.我認為目前為止只有兩種真正乾淨利落, 
始終如一的編程模式: C 語言模式和Lisp語言模式.此二者就像兩座高地, 
在它們中間是猶如沼澤的低地.隨著計算機變得越來越強大, 新開發的語言一直在堅定地 
趨向於Lisp模式. 
二十年來, 開發新編程語言的一個流行的秘決是, 取C語言的計 
算模式, 逐漸地往上加Lisp模式的特性, 例如運行時類型和無用單元收集. 

在這篇文章中我儘可能用最簡單的術語來解釋約翰麥卡錫所做的發現. 
關鍵是我們不僅要學習某個人四十年前得出的有趣理論結果, 
而且展示編程語言的發展方向. 
Lisp的不同尋常之處---也就是它優質的定義---是它能夠自己來編寫自己. 
為了理解約翰麥卡錫所表述的這個特點, 我們將追溯他的步伐, 並將他的數學標記 
轉換成能夠運行的 Common Lisp 代碼. 

\section{七個原始操作符} 
開始我們先定義{\em 
表達式}.表達式或是一個{\em 
原子}(atom), 它是一個字母序列(如 
foo), 或是一個由零個或多個表達式組成的{\em 
表}(list), 
表達式之間用空格分開, 
放入一對括號中. 
以下是一些表達式: 
\begin{verbatim} 
foo 
() 
(foo) 
(foo bar) 
(a b (c) d) 
\end{verbatim} 
最後一個表達式是由四個元素組成的表, 
第三個元素本身是由一個元素組成的表. 

在算術中表達式 $1 + 1$ 得出值2. 
正確的Lisp表達式也有值. 
如果表達式{\it e}得出值{\it v}, 我們說{\it e}{\em 返回}{\it v}. 
下一步我們將定義幾種表達式以及它們的返回值. 

如果一個表達式是表, 我們稱第一個元素為{\em 操作符}, 其餘的元素為{\em 自變量}.我們將定義七個原始
(從公理的意義上說)操作符: 
quote, atom, eq, car, cdr, cons, 和 cond. 

\begin{enumerate} 
\item 
(quote 
{\it 
x}) 
返回{\it 
x}.為了可讀性我們把(quote 
{\it 
x})簡記 
為'{\it 
x}. 

\begin{verbatim} 
> 
(quote a) a 
> 'a 
a 
> (quote (a b c)) (a b c) 
\end{verbatim} 
\item 
(atom 
{\it 
x})返回原子t如果{\it 
x}的值是一個原子或是空表, 否則返回(). 
在Lisp中我們 
按慣例用原子t表示真, 
而用空表表示假. 
\begin{verbatim} 
> (atom 'a) t 
> (atom '(a b c)) () 
> (atom '()) t 
\end{verbatim} 
既然有了一個自變量需要求值的操作符, 
我們可以看一下quote的作用. 
通過引 
用(quote)一個表, 我們避免它被求值. 
一個未被引用的表作為自變量傳給象 
atom這樣的操作符將被視為代碼: 
\begin{verbatim} 
> (atom (atom 'a)) t 
\end{verbatim} 
反之一個被引用的表僅被視為表, 
在此例中就是有兩個元素的表: 
\begin{verbatim} 
> (atom '(atom 'a)) () 
\end{verbatim} 
這與我們在英語中使用引號的方式一致. 
{\rm 
Cambridge}(劍橋)是一個位於麻薩諸塞 
州有90000人口的城鎮. 
而``{\rm 
Cambridge}''是一個由9個字母組成的單詞. 
引用看上去可能有點奇怪因為極少有其它語言有類似的概念. 
它和Lisp最與眾 
不同的特徵緊密聯繫:代碼和數據由相同的數據結構構成, 
而我們用quote操作符 
來區分它們. 
\item 
(eq 
{\it 
x} 
{\it 
y})返回t如果{\it 
x}和{\it 
y}的值是同一個原子或都是空表, 
否則返回(). 
\begin{verbatim} 
> (eq 'a 'a) t 
> (eq 'a 'b) () 
> (eq '() '()) t 
\end{verbatim} 
\item 
(car 
{\it 
x})期望{\it 
x}的值是一個表並且返回{\it 
x}的第一個元素. 
\begin{verbatim} 
> (car '(a b c)) 
a 
\end{verbatim} 
\item 
(cdr {\it x})期望{\it x}的值是一個表並且返回 {\it x} 的第一個元素之後的所有元素. 
\begin{verbatim} 
> (cdr '(a b c)) 
(b c) 
\end{verbatim} 
\item 
(cons 
{\it 
x} 
{\it 
y})期望{\it 
y}的值是一個表並且返回一個新表, 它的第一個元素是{\it 
x}的值, 
後 
面跟著{\it 
y}的值的各個元素. 
\begin{verbatim} 
> (cons 'a '(b c)) 
(a b c) 
> (cons 'a (cons 'b (cons 'c '()))) 
(a b c) 
> (car (cons 'a '(b c))) 
a 
> (cdr (cons 'a '(b c))) 
(b c) 
\end{verbatim} 
\item 
(cond 
(\pone\dots\eone) 
\dots 
(\pn\dots\en)) 
的求值規則如下. 
{\it 
p}表達式依次求值直到有一個 
返回t. 
如果能找到這樣的{\it 
p}表達式, 相應的{\it 
e}表達式的值作為整個cond表達式的 
返回值. 
\begin{verbatim} 
> (cond ((eq 'a 'b) 'first) 
    ((atom 'a) 'second)) 
second 
\end{verbatim} 
當表達式以七個原始操作符中的五個開頭時, 它的自變量總是要求值的.\footnote{以另外兩個操作符quote和cond開頭的表達式以不同的方式求值. 
當 
quote表達式求值時, 
它的自變量不被求值, 而是作為整個表達式的值返回. 
在一個正確的cond表達式中, 
只有L形路徑上的子表達式會被求值.} 
我們稱這樣的操作符為{\em 
函數}. 
\end{enumerate} 
\section{函數的表示} 
接著我們定義一個記號來描述函數.函數表示為(lambda 
(\pone\dots\pn) 
{\it 
e}), 其中 
\pone\dots\pn是原子(叫做{\em 
參數}), {\it 
e}是表達式. 
如果表達式的第一個元素形式如上 
\noindent{\tt 
((lambda 
(\pone\dots\pn) 
{\it 
e}) 
\aone\dots\an) 
} 
\noindent則稱為{\em 
函數調用}.它的值計算如下.每一個表達式{$a_{i}$}先求值, 然後{\it 
e}再求值.在{\it 
e}的 
求值過程中, 每個出現在{\it 
e}中的{$p_{i}$}的值是相應的{$a_{i}$}在最近一 
次的函數調用中的值. 
\begin{verbatim} 
> ((lambda (x) (cons x '(b))) 'a) 
(a b) 
> ((lambda (x y) (cons x (cdr y))) 
  'z 
  '(a b c)) 
(z b c) 
\end{verbatim} 
如果一個表達式的第一個元素{\it 
f}是原子且{\it 
f}不是原始操作符 
\noindent{\tt 
(f 
\aone\dots\an) 
} 
\noindent並且{\it 
f}的值是一個函數(lambda 
(\pone\dots\pn)), 則以上表達式的值就是 
\noindent{\tt 
((lambda 
(\pone\dots\pn) 
{\it 
e}) 
\aone\dots\an) 
} 
\noindent的值. 
換句話說, 參數在表達式中不但可以作為自變量也可以作為操作符使用: 
\begin{verbatim} 
> ((lambda (f) (f '(b c))) 
  '(lambda (x) (cons 'a x))) 
(a b c) 
\end{verbatim} 
有另外一個函數記號使得函數能提及它本身, 這樣我們就能方便地定義遞歸函 
數.\footnote{邏輯上我們不需要為了這定義一個新的記號. 
在現有的記號中用 
一個叫做Y組合器的函數上的函數, 
我們可以定義遞歸函數. 
可能麥卡錫在寫 
這篇論文的時候還不知道Y組合器; 
無論如何, 
label可讀性更強.} 
記號 
\noindent{\tt 
(label 
f 
(lambda 
(\pone\dots\pn) 
{\it 
e})) 
} 
\noindent表示一個象(lambda 
(\pone\dots\pn) 
{\it 
e})那樣的函數, 加上這樣的特性: 
任何出現在{\it 
e}中的{\it 
f}將求值為此label表達式, 
就好像{\it 
f}是此函數的參數. 
假設我們要定義函數(subst 
{\it 
x 
y 
z}), 
它取表達式{\it 
x}, 原子{\it 
y}和表{\it 
z}做參數, 返回一個 
象{\it 
z}那樣的表, 
不過{\it 
z}中出現的{\it 
y}(在任何嵌套層次上)被{\it 
x}代替. 
\begin{verbatim} 
> (subst 'm 'b '(a b (a b c) d)) (a m (a m c) d) 
\end{verbatim} 
我們可以這樣表示此函數 
\begin{verbatim} 
(label subst (lambda (x y z) (cond ((atom z) (cond ((eq z y) x) ('t z))) ('t (cons (subst x y (car z)) (subst x y (cdr z))))))) 
\end{verbatim} 
我們簡記{\it 
f}=(label 
{\it 
f} 
(lambda 
(\pone\dots\pn) 
{\it 
e}))為 
\noindent{\tt 
(defun 
{\it 
f} 
(\pone\dots\pn) 
{\it 
e}) 
} 
\noindent於是 
\begin{verbatim} 
(defun subst 
(x 
y 
z) 
(cond 
((atom 
z) 
(cond 
((eq 
z 
y) 
x) 
('t 
z))) 
('t 
(cons 
(subst 
x 
y 
(car 
z)) 
(subst 
x 
y 
(cdr 
z)))))) 
\end{verbatim} 
偶然地我們在這兒看到如何寫cond表達式的缺省子句. 
第一個元素是't的子句總 
是會成功的. 
於是 
\noindent{\tt 
(cond 
({\it 
x 
y}) 
('t 
{\it 
z})) 
} 
\noindent等同於我們在某些語言中寫的 
\noindent{\tt 
if 
{\it 
x} 
then 
{\it 
y} 
else 
{\it 
z} 
} 
\section{一些函數} 
既然我們有了表示函數的方法, 我們根據七個原始操作符來定義一些新的函數. 
為了方便我們引進一些常見模式的簡記法. 
我們用c{\it 
x}r, 其中{\it 
x}是a或d的序列, 來 
簡記相應的car和cdr的組合. 
比如(cadr 
{\it 
e})是(car 
(cdr 
{\it 
e}))的簡記, 它返回{\it 
e}的 
第二個元素. 
\begin{verbatim} 
> (cadr '((a b) (c d) e)) (c d) 
> (caddr '((a b) (c d) e)) e > (cdar '((a b) (c d) e)) (b) 
\end{verbatim} 
我們還用(list 
\eone\dots\en)表示(cons 
\eone\dots(cons 
\en 
'()) 
\dots). 
\begin{verbatim} 
> 
(cons 
'a 
(cons 
'b 
(cons 
'c 
'()))) 
(a 
b 
c) 
> 
(list 
'a 
'b 
'c) 
(a 
b 
c) 
\end{verbatim} 
現在我們定義一些新函數. 
我在函數名後面加了點, 以區別函數和定義它們的原 
始函數, 也避免與現存的common 
Lisp的函數衝突. 
\begin{enumerate} 
\item 
(null. 
{\it 
x})測試它的自變量是否是空表. 
\begin{verbatim} 
(defun 
null. 
(x) 
(eq 
x 
'())) 
> 
(null. 
'a) 
() 
> 
(null. 
'()) 
t 
\end{verbatim} 
\item 
(and. 
{\it 
x 
y})返回t如果它的兩個自變量都是t, 
否則返回(). 
\begin{verbatim} 
(defun 
and. 
(x 
y) 
(cond 
(x 
(cond 
(y 
't) 
('t 
'()))) 
('t 
'()))) 
> 
(and. 
(atom 
'a) 
(eq 
'a 
'a)) 
t 
> 
(and. 
(atom 
'a) 
(eq 
'a 
'b)) 
() 
\end{verbatim} 
\item 
(not. 
{\it 
x})返回t如果它的自變量返回(), 返回()如果它的自變量返回t. 
\begin{verbatim} 
(defun 
not. 
(x) 
(cond 
(x 
'()) 
('t 
't))) 
> 
(not. 
(eq 
'a 
'a)) 
() 
> 
(not. 
(eq 
'a 
'b)) 
t 
\end{verbatim} 
\item 
(append. 
{\t 
x 
y})取兩個表並返回它們的連結. 
\begin{verbatim} 
(defun 
append. 
(x 
y) 
(cond 
((null. 
x) 
y) 
('t 
(cons 
(car 
x) 
(append. 
(cdr 
x) 
y))))) 
> 
(append. 
'(a 
b) 
'(c 
d)) 
(a 
b 
c 
d) 
> 
(append. 
'() 
'(c 
d)) 
(c 
d) 
\end{verbatim} 
\item 
(pair. 
{\it 
x 
y})取兩個相同長度的表, 返回一個由雙元素表構成的表, 雙元素表是相 
應位置的x, y的元素對. 
\begin{verbatim} 
(defun 
pair. 
(x 
y) 
(cond 
((and. 
(null. 
x) 
(null. 
y)) 
'()) 
((and. 
(not. 
(atom 
x)) 
(not. 
(atom 
y))) 
(cons 
(list 
(car 
x) 
(car 
y)) 
(pair. 
(cdr) 
(cdr 
y)))))) 
> 
(pair. 
'(x 
y 
z) 
'(a 
b 
c)) 
((x 
a) 
(y 
b) 
(z 
c)) 
\end{verbatim} 
\item 
(assoc. 
{\it 
x 
y})取原子{\it 
x}和形如pair.函數所返回的表{\it 
y}, 返回{\it 
y}中第一個符合如下條 
件的表的第二個元素:它的第一個元素是{\it 
x}. 
\begin{verbatim} 
(defun 
assoc. 
(x 
y) 
(cond 
((eq 
(caar 
y) 
x) 
(cadar 
y)) 
('t 
(assoc. 
x 
(cdr 
y))))) 
> 
(assoc. 
'x 
'((x 
a) 
(y 
b))) 
a 
> 
(assoc. 
'x 
'((x 
new) 
(x 
a) 
(y 
b))) 
new 
\end{verbatim} 
\end{enumerate} 
\section{一個驚喜} 
因此我們能夠定義函數來連接表, 替換表達式等等.也許算是一個優美的表示法, 
那下一步呢? 
現在驚喜來了. 
我們可以寫一個函數作為我們語言的解釋器:此函 
數取任意Lisp表達式作自變量並返回它的值. 
如下所示: 
\begin{verbatim} 
(defun 
eval. 
(e 
a) 
(cond 
((atom 
e) 
(assoc. 
e 
a)) 
((atom 
(car 
e)) 
(cond 
((eq 
(car 
e) 
'quote) 
(cadr 
e)) 
((eq 
(car 
e) 
'atom) 
(atom 
(eval. 
(cadr 
e) 
a))) 
((eq 
(car 
e) 
'eq) 
(eq 
(eval. 
(cadr 
e) 
a) 
(eval. 
(caddr 
e) 
a))) 
((eq 
(car 
e) 
'car) 
(car 
(eval. 
(cadr 
e) 
a))) 
((eq 
(car 
e) 
'cdr) 
(cdr 
(eval. 
(cadr 
e) 
a))) 
((eq 
(car 
e) 
'cons) 
(cons 
(eval. 
(cadr 
e) 
a) 
(eval. 
(caddr 
e) 
a))) 
((eq 
(car 
e) 
'cond) 
(evcon. 
(cdr 
e) 
a)) 
('t 
(eval. 
(cons 
(assoc. 
(car 
e) 
a) 
(cdr 
e)) 
a)))) 
((eq 
(caar 
e) 
'label) 
(eval. 
(cons 
(caddar 
e) 
(cdr 
e)) 
(cons 
(list 
(cadar 
e) 
(car 
e)) 
a))) 
((eq 
(caar 
e) 
'lambda) 
(eval. 
(caddar 
e) 
(append. 
(pair. 
(cadar 
e) 
(evlis. 
(cdr 
e) 
a)) 
a))))) 
(defun 
evcon. 
(c 
a) 
(cond 
((eval. 
(caar 
c) 
a) 
(eval. 
(cadar 
c) 
a)) 
('t 
(evcon. 
(cdr 
c) 
a)))) 
(defun 
evlis. 
(m 
a) 
(cond 
((null. 
m) 
'()) 
('t 
(cons 
(eval. 
(car 
m) 
a) 
(evlis. 
(cdr 
m) 
a))))) 
\end{verbatim} 
eval.的定義比我們以前看到的都要長. 
讓我們考慮它的每一部分是如何工作的. 
eval.有兩個自變量: 
e是要求值的表達式, 
a是由一些賦給原子的值構成的表, 這 
些值有點象函數調用中的參數. 
這個形如pair.的返回值的表叫做{\em 
環境}. 
正是 
為了構造和搜索這種表我們才寫了pair.和assoc.. 
eval.的骨架是一個有四個子句的cond表達式. 
如何對表達式求值取決於它的類 
型. 
第一個子句處理原子. 
如果e是原子, 
我們在環境中尋找它的值: 
\begin{verbatim} 
> 
(eval. 
'x 
'((x 
a) 
(y 
b))) 
a 
\end{verbatim} 
第二個子句是另一個cond, 
它處理形如({\it 
a} 
\dots)的表達式, 
其中{\it 
a}是原子. 
這包 
括所有的原始操作符, 
每個對應一條子句. 
\begin{verbatim} 
> 
(eval. 
'(eq 
'a 
'a) 
'()) 
t 
> 
(eval. 
'(cons 
x 
'(b 
c)) 
'((x 
a) 
(y 
b))) 
(a 
b 
c) 
\end{verbatim} 
這幾個子句(除了quote)都調用eval.來尋找自變量的值. 
最後兩個子句更複雜些. 
為了求cond表達式的值我們調用了一個叫 
evcon.的輔助函數. 
它遞歸地對cond子句進行求值, 尋找第一個元素返回t的子句. 
如果找到 
了這樣的子句, 
它返回此子句的第二個元素. 
\begin{verbatim} 
> 
(eval. 
'(cond 
((atom 
x) 
'atom) 
('t 
'list)) 
'((x 
'(a 
b)))) 
list 
\end{verbatim} 
第二個子句的最後部分處理函數調用. 
它把原子替換為它的值(應該是lambda 
或label表達式)然後對所得結果表達式求值. 
於是 
\begin{verbatim} 
(eval. 
'(f 
'(b 
c)) 
'((f 
(lambda 
(x) 
(cons 
'a 
x))))) 
\end{verbatim} 
變為 
\begin{verbatim} 
(eval. 
'((lambda 
(x) 
(cons 
'a 
x)) 
'(b 
c)) 
'((f 
(lambda 
(x) 
(cons 
'a 
x))))) 
\end{verbatim} 
它返回(a 
b 
c). 
eval.的最後cond兩個子句處理第一個元素是lambda或label的函數調用.為了對label 
表達式求值, 
先把函數名和函數本身壓入環境, 
然後調用eval.對一個內部有 
lambda的表達式求值. 
即: 
\begin{verbatim} 
(eval. 
'((label 
firstatom 
(lambda 
(x) 
(cond 
((atom 
x) 
x) 
('t 
(firstatom 
(car 
x)))))) 
y) 
'((y 
((a 
b) 
(c 
d))))) 
\end{verbatim} 
變為 
\begin{verbatim} 
(eval. 
'((lambda 
(x) 
(cond 
((atom 
x) 
x) 
('t 
(firstatom 
(car 
x))))) 
y) 
'((firstatom 
(label 
firstatom 
(lambda 
(x) 
(cond 
((atom 
x) 
x) 
('t 
(firstatom 
(car 
x))))))) 
(y 
((a 
b) 
(c 
d))))) 
\end{verbatim} 
最終返回a. 
最後, 對形如((lambda 
(\pone\dots\pn) 
{\it 
e}) 
\aone\dots\an)的表達式求值, 先調用evlis.來 
求得自變量(\aone\dots\an)對應的值(\vone\dots\vn), 把(\pone 
\vone)\dots(\pn 
\vn)添加到 
環境裡, 
然後對{\it 
e}求值. 
於是 
\begin{verbatim} 
(eval. 
'((lambda 
(x 
y) 
(cons 
x 
(cdr 
y))) 
'a 
'(b 
c 
d)) 
'()) 
\end{verbatim} 
變為 
\begin{verbatim} 
(eval. 
'(cons 
x 
(cdr 
y)) 
'((x 
a) 
(y 
(b 
c 
d)))) 
\end{verbatim} 
最終返回(a 
c 
d). 
\section{後果} 
既然理解了eval是如何工作的, 
讓我們回過頭考慮一下這意味著什麼. 
我們在這 
兒得到了一個非常優美的計算模型. 
僅用quote, atom, eq, car, cdr, cons, 和cond, 
我們定義了函數eval., 它事實上實現了我們的語言, 用它可以定義任何我們想要 
的額外的函數. 
當然早已有了各種計算模型---最著名的是圖靈機. 
但是圖靈機程序難以讀懂. 
如果你要一種描述算法的語言, 
你可能需要更抽象的, 
而這就是約翰麥卡錫定義 
Lisp的目標之一. 
約翰麥卡錫於1960年定義的語言還缺不少東西. 
它沒有副作用, 
沒有連續執行 
(它得和副作用在一起才有用), 
沒有實際可用的數, \footnote{在麥卡錫的1960 
年的Lisp中, 
做算術是可能的, 
比如用一個有n個原子的表表示數n.} 
沒有動態可視域. 
但這些限制可 
以令人驚訝地用極少的額外代碼來補救. 
Steele和Sussman在一篇叫做``解釋器 
的藝術''的著名論文中描述了如何做到這點.\footnote{Guy 
Lewis 
Steele, 
Jr. 
and 
Gerald 
Jay 
Sussman, 
``The 
Art 
of 
the 
Interpreter, 
or 
the 
Modularity 
Complex(Parts 
Zero, One, and 
Two), '' 
MIT 
AL 
Lab 
Memo 
453, 
May 
1978.} 
如果你理解了約翰麥卡錫的eval, 
那你就不僅僅是理解了程序語言歷史中的一個 
階段. 
這些思想至今仍是Lisp的語義核心. 
所以從某種意義上, 
學習約翰麥卡 
錫的原著向我們展示了Lisp究竟是什麼. 
與其說Lisp是麥卡錫的設計, 不如說是 
他的發現. 
它不是生來就是一門用於人工智能, 
快速原型開發或同等層次任務的 
語言. 
它是你試圖公理化計算的結果(之一). 
隨著時間的推移, 
中級語言, 
即被中間層程序員使用的語言, 
正一致地向Lisp靠 
近. 
因此通過理解eval你正在明白將來的主流計算模式會是什麼樣. 
\section{註釋} 
把約翰麥卡錫的記號翻譯為代碼的過程中我儘可能地少做改動. 
我有過讓代碼 
更容易閱讀的念頭, 
但是我還是想保持原汁原味. 
在約翰麥卡錫的論文中, 假用f來表示, 
而不是空表. 
我用空表表示假以使例子能 
在Common 
Lisp中運行. 
(fixme) 
我略過了構造dotted 
pairs, 
因為你不需要它來理解eval. 
我也沒有提apply, 
雖然是apply(它的早期形式, 
主要作用是引用自變量), 
被約翰麥卡錫在1960年 
稱為普遍函數, 
eval只是不過是被apply調用的子程序來完成所有的工作. 
我定義了list和c{\it 
x}r等作為簡記法因為麥卡錫就是這麼做的. 
實際上 
c{\it 
x}r等可以 
被定義為普通的函數. 
List也可以這樣, 
如果我們修改eval, 
這很容易做到, 
讓 
函數可以接受任意數目的自變量. 
麥卡錫的論文中只有五個原始操作符. 
他使用了cond和quote, 但可能把它們作 
為他的元語言的一部分. 
同樣他也沒有定義邏輯操作符and和not, 
這不是個問題, 
因為它們可以被定義成合適的函數. 
在eval.的定義中我們調用了其它函數如pair.和assoc., 但任何我們用原始操作 
符定義的函數調用都可以用eval.來代替. 
即 
\begin{verbatim} 
(assoc. 
(car 
e) 
a) 
\end{verbatim} 
能寫成 
\begin{verbatim} 
(eval. 
'((label 
assoc. 
(lambda 
(x 
y) 
(cond 
((eq 
(caar 
y) 
x) 
(cadar 
y)) 
('t 
(assoc. 
x 
(cdr 
y)))))) 
(car 
e) 
a) 
(cons 
(list 
'e 
e) 
(cons 
(list 
'a 
a) 
a))) 
\end{verbatim} 
麥卡錫的eval有一個錯誤. 
第16行是(相當於)(evlis. 
(cdr 
e) 
a)而不是(cdr 
e), 
這使得自變量在一個有名函數的調用中被求值兩次. 
這顯示當論文發表的 
時候, 
eval的這種描述還沒有用IBM 
704機器語言實現. 
它還證明了如果不去運 
行程序, 
要保證不管多短的程序的正確性是多麼困難. 
我還在麥卡錫的論文中碰到一個問題. 
在定義了eval之後, 
他繼續給出了一些 
更高級的函數---接受其它函數作為自變量的函數. 
他定義了maplist: 
\begin{verbatim} 
(label 
maplist 
(lambda 
(x 
f) 
(cond 
((null 
x) 
'()) 
('t 
(cons 
(f 
x) 
(maplist 
(cdr 
x) 
f)))))) 
\end{verbatim} 
然後用它寫了一個做微分的簡單函數diff. 
但是diff傳給maplist一個用{\it 
x}做參 
數的函數, 
對它的引用被maplist中的參數x所捕獲.\footnote{當代的Lisp程序 
員在這兒會用mapcar代替maplist. 
這個例子解開了一個謎團: 
maplist為什 
麼會在Common 
Lisp中. 
它是最早的映射函數, 
mapcar是後來增加的.} 
這是關於動態可視域危險性的雄辯證據, 
即使是最早的更高級函數的例子也因為 
它而出錯. 
可能麥卡錫在1960年還沒有充分意識到動態可視域的含意. 
動態可 
視域令人驚異地在Lisp實現中存在了相當長的時間---直到Sussman和Steele於 
1975年開發了Scheme. 
詞法可視域沒使eval的定義複雜多少, 
卻使編譯器更難 
寫了. 
%\end{comment}
\newpage 
\end{CJK} 
\end{document}
