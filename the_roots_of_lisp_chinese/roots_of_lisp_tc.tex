% http://daiyuwen.freeshell.org/gb/rol/roots_of_lisp.tex.gz
%\documentclass[12pt, a4]{article}
\documentclass[12pt]{article}
\usepackage[paperwidth=10.1cm, paperheight=13cm, left=0.3cm, right=0.3cm, bottom=0.5cm, top=0.1cm, nohead, nofoot]{geometry}

\parindent=0cm
\parskip=13pt

\usepackage{CJKutf8} %使用CJK套件
\usepackage{comment}

\usepackage{hyperref} % 最好保证 hyperref 是最后加载的宏包
\hypersetup{%
  dvipdfmx,% 设定要使用的 driver 为 dvipdfmx
  unicode={true},% 使用 unicode 来编码 PDF 字符串
  pdfstartview={FitH},% 文档初始视图为匹配宽度
  bookmarksnumbered={true},% 书签附上章节编号
  bookmarksopen={true},% 展开书签
  pdfborder={0 0 0},% 链接无框
  pdftitle={The Roots of Lisp 中文版本},
  pdfauthor={Paul Graham},
  pdfsubject={The Roots of Lisp 中文版本},
  pdfkeywords={lisp},
  pdfcreator={应用程序},
  pdfproducer={PDF 制作程序},% 这个好像没起作用?
}

\begin{document} 
\begin{CJK}{UTF8}{bsmi} %開始CJK環境,設定編碼,設定字體
%\CJKtilde 
%\CJKindent 
%\CJKcaption{GB} 
\title{Lisp 之根源 (The Roots of Lisp)} 
\author{保羅格雷厄姆 (Paul Graham)} 
\maketitle 
%\begin{comment}
% some abbreviations 
\newcommand{\pone}{$p_{1}$} 
\newcommand{\pn}{$p_{n}$} 
\newcommand{\aone}{$a_{1}$} 
\newcommand{\an}{$a_{n}$} 
\newcommand{\vone}{$v_{1}$} 
\newcommand{\vn}{$v_{n}$} 
\newcommand{\eone}{$e_{1}$} 
\newcommand{\en}{$e_{n}$} 
約翰麥卡錫於1960年發表了一篇非凡的論文,
他在這篇論文中對編程的貢獻有如歐幾里德對幾何的貢獻.\footnote{``Recursive
Functions of Symbolic Expressions and Their Computation by Machine, Part1.'' 
{\it Communication of the ACM} 3:4, April 1960, pp. 184-195.} 
他向我們展示了, 在只給定幾個簡單的操作符和一個表示函數的記號的基礎上, 
如何構造出一個完整的編程語言. 
麥卡錫稱這種語言為Lisp, 
意為 
List 
Processing, 
因為他的主要思想之一是用一種簡單的資料結構表(list)來代表代碼和資料. 

值得注意的是, 麥卡錫所作的發現, 不僅是計算機史上劃時代的大事, 
而且是一種在我們這個時代編程越來越趨向的模式.我認為目前為止只有兩種真正乾淨利落, 
始終如一的編程模式: C 語言模式和Lisp語言模式.此二者就像兩座高地, 
在它們中間是猶如沼澤的低地.隨著計算機變得越來越強大, 新開發的語言一直在堅定地 
趨向於Lisp模式. 
二十年來, 開發新編程語言的一個流行的秘決是, 取C語言的計 
算模式, 逐漸地往上加Lisp模式的特性, 例如運行時類型和無用單元收集. 

在這篇文章中我儘可能用最簡單的術語來解釋約翰麥卡錫所做的發現. 
關鍵是我們不僅要學習某個人四十年前得出的有趣理論結果, 
而且展示編程語言的發展方向. 
Lisp的不同尋常之處---也就是它優質的定義---是它能夠自己來編寫自己. 
為了理解約翰麥卡錫所表述的這個特點, 我們將追溯他的步伐, 並將他的數學標記 
轉換成能夠運行的 Common Lisp 代碼. 

\section{七個 primitive operators} 
開始我們先定義{\em 表達式 (expression)}。表達式要麼是一個{\em 原子}(atom),
它是一個字母序列 (如 foo), 要麼是由零個或多個表達式組成的{\em 表}(list), 
表達式之間用空格分開, 放入一對括號中。以下是一些表達式: 
\begin{verbatim} 
foo 
() 
(foo) 
(foo bar) 
(a b (c) d) 
\end{verbatim} 
最後一個表達式是由四個元素組成的表, 
第三個元素本身是由一個元素組成的表. 

在算術中表達式 $1 + 1$ 得出值2. 
正確的Lisp表達式也有值. 
如果表達式{\it e}得出值{\it v}, 我們說{\it e}{\em 返回}{\it v}. 
下一步我們將定義幾種表達式以及它們的返回值. 

如果一個表達式是表, 我們稱第一個元素為{\em operator}, 其餘的元素為{\em 引數}.我們將定義七個原始
(從公理的意義上說)操作符: 
quote, atom, eq, car, cdr, cons, 和 cond. 

\begin{enumerate} 
\item 
(quote 
{\it 
x}) 
返回{\it 
x}.為了可讀性我們把(quote 
{\it 
x})簡記 
為'{\it 
x}. 

\begin{verbatim} 
> 
(quote a) a 
> 'a 
a 
> (quote (a b c)) (a b c) 
\end{verbatim} 
\item 
(atom 
{\it 
x})返回原子t如果{\it 
x}的值是一個原子或是空表, 否則返回(). 
在Lisp中我們 
按慣例用原子t表示真, 
而用空表表示假. 
\begin{verbatim} 
> (atom 'a) t 
> (atom '(a b c)) () 
> (atom '()) t 
\end{verbatim} 
既然有了一個引數需要求值的操作符, 
我們可以看一下 quote 的作用. 
通過引用 (quote) 一個表, 我們避免它被求值. 
一個未被引用的表作為參數傳給 atom 這樣的操作符將被視為代碼 (code): 
\begin{verbatim} 
> (atom (atom 'a)) t 
\end{verbatim} 
反之一個被引用的表僅被視為表, 
在此例中就是有兩個元素的表: 
\begin{verbatim} 
> (atom '(atom 'a)) () 
\end{verbatim} 
這與我們在英語中使用引號的方式一致. 
{\rm Cambridge} (劍橋) 是一個位於麻薩諸塞州有 90000 人口的城鎮. 
而``{\rm Cambridge}''是一個由9個字母組成的單詞. 
引用看上去可能有點奇怪因為極少有其它語言有類似的概念. 
它和Lisp最與眾不同的特徵緊密聯繫: 代碼和資料由相同的資料結構構成, 
而我們用quote操作符來區分它們. 
\item 
(eq {\it x} {\it y})返回t如果{\it x}和{\it y}的值是同一個原子或都是空表, 否則返回() (descent 注: () 表示 false). 
\begin{verbatim} 
> (eq 'a 'a) 
t 
> (eq 'a 'b) 
() 
> (eq '() '()) 
t 
\end{verbatim} 
\item 
(car {\it x})期望{\it x}的值是一個表並且返回{\it x}的第一個元素. 
\begin{verbatim} 
> (car '(a b c)) 
a 
\end{verbatim} 
\item 
(cdr {\it x})期望{\it x}的值是一個表並且返回 {\it x} 的第一個元素之後的所有元素. 
\begin{verbatim} 
> (cdr '(a b c)) 
(b c) 
\end{verbatim} 
\item 
(cons {\it x} {\it y})期望{\it y}的值是一個表並且返回一個新表, 它的第一個元素是{\it x}的值,
後面跟著{\it y}的值的各個元素. 
\begin{verbatim} 
> (cons 'a '(b c)) 
(a b c) 
> (cons 'a (cons 'b (cons 'c '()))) 
(a b c) 
> (car (cons 'a '(b c))) 
a 
> (cdr (cons 'a '(b c))) 
(b c) 
\end{verbatim} 
\item 
(cond 
(\pone\dots\eone) 
\dots 
(\pn\dots\en)) 
的求值規則如下. 
{\it p}表達式依次求值直到有一個返回t. 
如果能找到這樣的{\it 
p}表達式, 相應的{\it 
e}表達式的值作為整個cond表達式的返回值. 
\begin{verbatim} 
> (cond ((eq 'a 'b) 'first) 
    ((atom 'a) 'second)) 
second 
\end{verbatim} 
這 7 個 primitive operator 其中的 5 個, 當一個表達式是以該操作符作為開頭被求值時, 他們的參數也會被求值
(In five of our seven primitive operators, the arguments are always evaluated 
when an expression beginning with that operator is evaluated.)。
\footnote{以另外兩個操作符 quote 和 cond 開頭的表達式以不同的方式求值.
當 quote 表達式求值時, 它的參數不被求值, 而是作為整個表達式的值返回. 
在一個正確的 cond 表達式中, 只有 L 形路徑上的子表達式會被求值
(And in a valid cond expression, only an L-shaped path of subexpressions will be evaluated.)。} 
我們稱這樣的操作符為{\em 函數}. 
\end{enumerate} 

\section{函數的表示} 

接著我們定義一個用來描述函數的記號。函數表示為
(lambda (\pone\dots\pn) {\it e}),
其中 \pone\dots\pn 是 atom (叫做{\em 參數 (parameters)}),
{\it e} 是表達式。如果表達式的第一個元素是像這樣的表達式\\
{\tt ((lambda (\pone\dots\pn) {\it e}) \aone\dots\an)} \\
則稱為{\em 函數呼叫 (funcation call)}。它的值計算如下。每一個表達式
{$a_{i}$} 先求值, 然後 {\it e} 再求值。在 {\it e}
的求值過程中, 每個出現在 {\it e} 中的 {$p_{i}$} 的值是相應的
{$a_{i}$} 在最近一次函數呼叫中的值。 

is called a function call and its value is computed as follows. Each expression
{$a_{i}$} is evaluated. Then e is evaluated. During the evaluation of e, the value of any occurrence of one of the {$p_{i}$} is the value of the corresponding {$a_{i}$} in the most
recent function call.

\begin{verbatim} 
> ((lambda (x) (cons x '(b))) 'a) 
(a b) 
> ((lambda (x y) (cons x (cdr y))) 
   'z 
   '(a b c)) 
(z b c) 
\end{verbatim} 
如果一個表達式的第一個元素 {\it f} 是 atom 且 {\it f} 不是 primitive
operators\\
{\tt (f \aone\dots\an) }\\ 

and the value of {\it f} is a function (lambda (\pone\dots\pn) {\it e})
then the value of the expression is the value of

並且 {\it f} 的值是一個函數 (lambda (\pone\dots\pn) {\it e}),
則以上表達式的值就是\\
{\tt ((lambda (\pone\dots\pn) {\it e}) \aone\dots\an) }\\ 
的值。換句話說, 參數 (parameters) 在表達式中不但可以作為引數 (arguments)
\footnote{descent: parameters, arguments 可參考: http://msdn.microsoft.com/zh-tw/library/9kewt1b3.aspx}
也可以作為 operators 使用: 

In other words, parameters can be used as operators in expressions as well as
arguments:

\begin{verbatim} 
> ((lambda (f) (f '(b c))) 
   '(lambda (x) (cons 'a x))) 
(a b c) 
\end{verbatim} 
有另外一個函數記號使得函數能提及它本身, 這樣我們就能方便地定義遞歸函數.
\footnote{邏輯上我們不需要為了這定義一個新的記號. 
在現有的記號中用一個叫做Y組合器的函數上的函數, 
我們可以定義遞歸函數. 
可能麥卡錫在寫這篇論文的時候還不知道 Y 組合器; 
無論如何, label 可讀性更強.} 
記號\\ 
{\tt (label f (lambda (\pone\dots\pn) {\it e})) }\\
\noindent表示一個像 (lambda (\pone\dots\pn) {\it e}) 那樣的函數,
加上這樣的特性: 
任何出現在{\it e}中的{\it f}將求值為此label表達式, 就好像{\it f}是此函數的參數. 
假設我們要定義函數 (subst {\it x y z}), 
它取表達式{\it x}, 原子{\it y}和表{\it z}做參數, 返回一個像 {\it z} 那樣的表, 
不過{\it z}中出現的{\it y}(在任何嵌套層次上)被{\it x}代替. 
\begin{verbatim} 
> (subst 'm 'b '(a b (a b c) d)) 
(a m (a m c) d) 
\end{verbatim} 
我們可以這樣表示此函數 
\begin{verbatim} 
(label subst (lambda (x y z) 
               (cond ((atom z) 
                      (cond ((eq z y) x) 
                            ('t z))) 
                     ('t (cons (subst x y (car z)) 
                               (subst x y (cdr z))))))) 
\end{verbatim} 
我們簡記 {\it f}=(label {\it f} (lambda (\pone\dots\pn) {\it e}))
為\\ 
{\tt (defun {\it f} (\pone\dots\pn) {\it e}) } \\
於是 
\begin{verbatim} 
(defun subst (x y z) 
  (cond ((atom z) 
         (cond ((eq z y) x) 
               ('t z))) 
        ('t (cons (subst x y (car z)) 
                  (subst x y (cdr z)))))) 
\end{verbatim} 
偶然地我們在這兒看到如何寫cond表達式的缺省子句. 
第一個元素是't的子句總 
是會成功的. 
於是 
\noindent{\tt 
(cond 
({\it 
x 
y}) 
('t 
{\it 
z})) 
} 
\noindent等同於我們在某些語言中寫的 
\noindent{\tt 
if 
{\it 
x} 
then 
{\it 
y} 
else 
{\it 
z} 
} 
\section{一些函數} 
既然我們有了表示函數的方法, 我們可以根據這七個 primitive operators 來定義一些新的函數。

為了方便我們引進一些常見模式的簡記法。
我們用 c{\it x}r, 其中 {\it x} 是 a 或 d 的序列, 來簡記相應的 car 和 cdr 的組合. 
比如 (cadr {\it e}) 是 (car (cdr {\it e})) 的簡記, 它返回 {\it e} 的第二個元素. 
\begin{verbatim} 
> (cadr '((a b) (c d) e)) 
(c d) 
> (caddr '((a b) (c d) e)) 
e 
> (cdar '((a b) (c d) e)) 
(b) 
\end{verbatim} 
我們還用 (list \eone\dots\en) 表示 (cons \eone\dots(cons \en '()) \dots)。
\begin{verbatim} 
> (cons 'a (cons 'b (cons 'c '()))) 
(a b c) 
> (list 'a 'b 'c) 
(a b c) 
\end{verbatim} 
現在我們定義一些新函數. 
我在函數名後面加了點, 以區別函數和定義它們的原始函數, 也避免與現存的 common 
Lisp 的函數衝突. 
\begin{enumerate} 
\item (null.  {\it x})測試它的引數是否是空表. 
\begin{verbatim} 
(defun null.  (x) 
  (eq x '())) 
> (null.  'a) 
() 
> (null.  '()) 
t 
\end{verbatim} 
\item 
(and. 
{\it 
x 
y})返回t如果它的兩個引數都是t, 
否則返回(). 
\begin{verbatim} 
(defun and.  (x y) 
  (cond (x (cond (y 't) ('t '()))) 
        ('t '()))) 
> (and.  (atom 'a) (eq 'a 'a)) 
t 
> (and.  (atom 'a) (eq 'a 'b)) 
() 
\end{verbatim} 
\item 
(not. 
{\it 
x})返回t如果它的引數返回(), 返回()如果它的引數返回t. 
\begin{verbatim} 
(defun not.  (x) 
  (cond (x '()) 
        ('t 't))) 
> (not.  (eq 'a 'a)) 
() 
> (not.  (eq 'a 'b)) 
t 
\end{verbatim} 
\item 
(append. 
{\t 
x 
y})取兩個表並返回它們的連結. 
\begin{verbatim} 
(defun append.  (x y) 
  (cond ((null.  x) y) 
    ('t (cons (car x) (append.  (cdr x) y))))) 
> (append.  '(a b) '(c d)) 
(a b c d) 
> (append.  '() '(c d)) 
(c d) 
\end{verbatim} 
\item 
(pair. 
{\it 
x 
y})取兩個相同長度的表, 返回一個由雙元素表構成的表, 雙元素表是相 
應位置的x, y的元素對. 
\begin{verbatim} 
(defun pair.  (x y) 
(cond ((and.  (null.  x) (null.  y)) '()) 
  ((and.  (not.  (atom x)) (not.  (atom y))) 
    (cons (list (car x) (car y)) (pair.  (cdr) (cdr y)))))) 
> (pair.  '(x y z) '(a b c)) 
((x a) (y b) (z c)) 
\end{verbatim} 
\item 
(assoc. 
{\it 
x 
y})取原子{\it 
x}和形如pair.函數所返回的表{\it 
y}, 返回{\it 
y}中第一個符合如下條 
件的表的第二個元素:它的第一個元素是{\it 
x}. 
\begin{verbatim} 
(defun assoc.  (x y) 
  (cond ((eq (caar y) x) (cadar y)) 
    ('t (assoc.  x (cdr y))))) 
> (assoc.  'x '((x a) (y b))) 
a 
> (assoc.  'x '((x new) (x a) (y b))) 
new 
\end{verbatim} 
\end{enumerate} 
\section{一個驚喜} 
因此我們能夠定義函數來連接表, 替換表達式等等.也許算是一個優美的表示法, 
那下一步呢? 
現在驚喜來了. 
我們可以寫一個函數作為我們語言的 interpreter : 此函數取任意Lisp表達式作引數並返回它的值. 
如下所示: 
\begin{verbatim} 
(defun eval.  (e a) 
  (cond 
    ((atom e) (assoc.  e a)) 
    ((atom (car e)) 
      (cond 
        ((eq (car e) 'quote) (cadr e)) 
        ((eq (car e) 'atom) (atom (eval.  (cadr e) a))) 
        ((eq (car e) 'eq) (eq (eval.  (cadr e) a) 
                              (eval.  (caddr e) a))) 
        ((eq (car e) 'car) (car (eval.  (cadr e) a))) 
        ((eq (car e) 'cdr) (cdr (eval.  (cadr e) a))) 
        ((eq (car e) 'cons) (cons (eval.  (cadr e) a) 
                                  (eval.  (caddr e) a))) 
        ((eq (car e) 'cond) (evcon.  (cdr e) a)) 
        ('t (eval.  (cons (assoc.  (car e) a) 
                          (cdr e)) 
                    a)))) 
      ((eq (caar e) 'label) 
       (eval.  (cons (caddar e) (cdr e)) 
               (cons (list (cadar e) (car e)) a))) 
      ((eq (caar e) 'lambda) 
       (eval.  (caddar e) 
               (append.  (pair.  (cadar e) (evlis.  (cdr e) a)) 
                         a))))) 
(defun evcon.  (c a) 
  (cond ((eval.  (caar c) a) 
         (eval.  (cadar c) a)) 
        ('t (evcon.  (cdr c) a)))) 
(defun evlis.  (m a) 
  (cond ((null.  m) '()) 
        ('t (cons (eval.  (car m) a) 
                  (evlis.  (cdr m) a))))) 
\end{verbatim} 
eval.的定義比我們以前看到的都要長. 
讓我們考慮它的每一部分是如何工作的. 
eval.有兩個引數: 
e是要求值的表達式, 
a是由一些賦給原子的值構成的表, 這 
些值有點象函數調用中的參數. 
這個形如pair.的返回值的表叫做{\em 
環境}. 
正是 
為了構造和搜索這種表我們才寫了pair.和assoc.. 
eval.的骨架是一個有四個子句的cond表達式. 
如何對表達式求值取決於它的類 
型. 
第一個子句處理原子. 
如果e是原子, 
我們在環境中尋找它的值: 
\begin{verbatim} 
> (eval.  'x '((x a) (y b))) 
a 
\end{verbatim} 
第二個子句是另一個cond, 
它處理形如({\it 
a} 
\dots)的表達式, 
其中{\it 
a}是原子. 
這包 
括所有的 primitive operators, 
每個對應一條子句. 
\begin{verbatim} 
> (eval.  '(eq 'a 'a) '()) 
t 
> (eval.  '(cons x '(b c)) 
          '((x a) (y b))) 
(a b c) 
\end{verbatim} 
這幾個子句(除了quote)都調用eval.來尋找引數的值. 
最後兩個子句更複雜些. 
為了求cond表達式的值我們調用了一個叫 
evcon.的輔助函數. 
它遞歸地對cond子句進行求值, 尋找第一個元素返回t的子句. 
如果找到 
了這樣的子句, 
它返回此子句的第二個元素. 
\begin{verbatim} 
> (eval.  '(cond ((atom x) 'atom) 
                 ('t 'list)) 
          '((x '(a b)))) 
list 
\end{verbatim} 
第二個子句的最後部分處理函數調用. 
它把原子替換為它的值(應該是lambda 
或label表達式)然後對所得結果表達式求值. 
於是 
\begin{verbatim} 
(eval.  '(f '(b c)) 
        '((f (lambda (x) (cons 'a x))))) 
\end{verbatim} 
變為 
\begin{verbatim} 
(eval.  '((lambda (x) (cons 'a x)) '(b c)) 
        '((f (lambda (x) (cons 'a x))))) 
\end{verbatim} 
它返回(a b c). 
eval.的最後cond兩個子句處理第一個元素是lambda或label的函數調用.為了對label 
表達式求值, 
先把函數名和函數本身壓入環境, 
然後調用eval.對一個內部有 
lambda的表達式求值. 
即: 
\begin{verbatim} 
(eval.  '((label firstatom (lambda (x) 
                             (cond ((atom x) x) 
                                   ('t (firstatom (car x)))))) 
          y) 
        '((y ((a b) (c d))))) 
\end{verbatim} 
變為 
\begin{verbatim} 
(eval.  '((lambda (x) 
            (cond ((atom x) x) 
                  ('t (firstatom (car x))))) 
         y) 
        '((firstatom 
           (label firstatom (lambda (x) 
                              (cond ((atom x) x) 
                                    ('t (firstatom (car x))))))) 
          (y ((a b) (c d))))) 
\end{verbatim} 
最終返回a. 
最後, 對形如((lambda 
(\pone\dots\pn) 
{\it 
e}) 
\aone\dots\an)的表達式求值, 先調用evlis.來 
求得引數(\aone\dots\an)對應的值(\vone\dots\vn), 把(\pone 
\vone)\dots(\pn 
\vn)添加到 
環境裡, 
然後對{\it 
e}求值. 
於是 
\begin{verbatim} 
(eval.  '((lambda (x y) (cons x (cdr y))) 
          'a 
          '(b c d)) 
        '()) 
\end{verbatim} 
變為 
\begin{verbatim} 
(eval.  '(cons x (cdr y)) 
        '((x a) (y (b c d)))) 
\end{verbatim} 
最終返回(a 
c 
d). 
\section{後果} 
既然理解了eval是如何工作的, 
讓我們回過頭考慮一下這意味著什麼. 
我們在這兒得到了一個非常優美的計算模型. 
僅用quote, atom, eq, car, cdr, cons, 和cond, 
我們定義了函數eval., 它事實上實現了我們的語言, 用它可以定義任何我們想要的額外的函數. 

當然早已有了各種計算模型---最著名的是圖靈機. 
但是圖靈機程序難以讀懂. 
如果你要一種描述算法的語言, 
你可能需要更抽象的, 
而這就是約翰麥卡錫定義 
Lisp的目標之一. 

約翰麥卡錫於1960年定義的語言還缺不少東西. 
它沒有副作用, 
沒有連續執行 
(它得和副作用在一起才有用), 
沒有實際可用的數, \footnote{在麥卡錫的1960 
年的Lisp中, 
做算術是可能的, 
比如用一個有n個原子的表表示數n.} 
沒有動態可視域. 
但這些限制可 
以令人驚訝地用極少的額外代碼來補救. 
Steele和Sussman
在一篇叫做``interpreter 的藝術''的著名論文中描述了如何做到這點.
\footnote{Guy Lewis Steele, Jr.  and Gerald Jay Sussman,
``The Art of the Interpreter, or the Modularity Complex(Parts Zero, One, and Two), '' 
MIT AL Lab Memo 453, May 1978.} 

如果你理解了約翰麥卡錫的 eval, 
那你就不僅僅是理解了程序語言歷史中的一個階段. 
這些思想至今仍是Lisp的語義核心. 
所以從某種意義上, 
學習約翰麥卡錫的原著向我們展示了Lisp究竟是什麼. 
與其說Lisp是麥卡錫的設計, 不如說是他的發現. 
它不是生來就是一門用於人工智能, 
快速原型開發或同等層次任務的語言. 
它是你試圖公理化計算的結果(之一). 

隨著時間的推移, 
中級語言, 
即被中間層程序員使用的語言, 
正一致地向Lisp靠近. 
因此通過理解eval你正在明白將來的主流計算模式會是什麼樣. 

\section{註釋} 
把約翰麥卡錫的記號翻譯為代碼的過程中我儘可能地少做改動. 
我有過讓代碼更容易閱讀的念頭, 
但是我還是想保持原汁原味. 

在約翰麥卡錫的論文中, 假用f來表示, 
而不是空表. 
我用空表表示假以使例子能在
Common Lisp 中運行. 

(fixme) 
我略過了構造dotted 
pairs, 
因為你不需要它來理解eval. 
我也沒有提apply, 
雖然是 apply (它的早期形式, 主要是引用 (quote) 引數), 在 1960 被約翰麥卡錫稱為普遍函數
(universal function), 
eval 只不過是被 apply 呼叫用的一個子程序藉以完成所有的工作. 

我定義了 list 和 c{\it x}r 等作為簡記法是因為麥卡錫就是這麼做的。
實際上 c{\it x}r等可以被定義為普通的函數 (ordinary functions)。List
也可以這樣, 如果我們修改 eval, 
這很容易做到, 讓函數可以接受任意數目的引數.  

麥卡錫的論文中只有五個 primitive operators. 
他使用了cond和quote, 但可能把它們作為他的元語言 (metalanguage) 的一部分. 
同樣他也沒有定義邏輯操作符and和not, 
這不是個問題, 
因為它們可以被定義成合適的函數. 

在eval.的定義中我們調用了其它函數如pair.和assoc.,
但任何我們用原始操作符定義的函數調用都可以用eval.來代替. 
即 
\begin{verbatim} 
(assoc.  (car e) a) 
\end{verbatim} 
能寫成 
\begin{verbatim} 
(eval.  '((label assoc. 
(lambda (x y) 
(cond ((eq (caar y) x) (cadar y)) 
('t (assoc.  x (cdr y)))))) 
(car e) 
a) 
(cons (list 'e e) (cons (list 'a a) a))) 
\end{verbatim} 
麥卡錫的eval有一個錯誤. 
第16行是(相當於)(evlis.  (cdr e) a)而不是(cdr e), 
這使得引數在一個有名函數的調用中被求值兩次. 
這顯示當論文發表的時候, 
eval 的這種描述還沒有用 IBM 704 機器語言實作。
它還證明了如果不去運行程序, 
要保證不管多短的程序的正確性是多麼困難. 

我還在麥卡錫的論文中碰到一個問題. 
在定義了eval之後, 
他繼續給出了一些更高級的函數---接受其它函數作為引數的函數. 
他定義了maplist: 
\begin{verbatim} 
(label maplist 
(lambda (x f) 
(cond ((null x) '()) 
('t (cons (f x) (maplist (cdr x) f)))))) 
\end{verbatim} 
然後用它寫了一個做微分的簡單函數diff. 
但是 diff 傳給 maplist 一個用 {\it x} 做參數的函數, 
對它的引用被 maplist 中的參數 x 所捕獲.
\footnote{當代的 Lisp 程序員在這兒會用 mapcar 代替 maplist. 
這個例子解開了一個謎團: 
maplist 為什麼會在 Common Lisp 中. 
它是最早的映射函數, mapcar 是後來增加的.} 

這是關於動態可視域危險性的雄辯證據, 
即使是最早的更高級函數的例子也因為它而出錯. 
可能麥卡錫在1960年還沒有充分意識到動態可視域的含意。動態可視域令人驚異地在
Lisp 實現中存在了相當長的時間---直到 Sussman 和 Steele 於 1975 年開發了 Scheme. 
詞法可視域沒使 eval 的定義複雜多少, 卻使編譯器更難寫了. 
%\end{comment}
\newpage 
\end{CJK} 
\end{document}
