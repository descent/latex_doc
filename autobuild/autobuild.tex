\documentclass[12pt,a4]{article}
\usepackage[paperwidth=10.1cm,paperheight=13cm,left=0.3cm,right=0.3cm,bottom=0.5cm,top=0.1cm,nohead,nofoot]{geometry}
% 轉成 PDF 時,產生 hyperlink 的效果
%\usepackage{CJK} %使用CJK套件
\usepackage{CJKutf8} %使用CJK套件
\usepackage{comment}
\usepackage{type1cm}
\usepackage{fancyvrb}
\usepackage{graphicx}
\usepackage{listings}
% 在使用 latex2html 可以處理中文
%\begin{htmlonly}
%\usepackage{taiwan}
%\end{htmlonly}
%\usepackage{cwtex}
\usepackage{html}
%\usepackage{appendix}

\usepackage{hyperref} % 最好保证 hyperref 是最后加载的宏包
\hypersetup{%
%  dvipdfmx,% 设定要使用的 driver 为 dvipdfmx
  dvips,% 设定要使用的 driver 为 dvipdfmx
  unicode={true},% 使用 unicode 来编码 PDF 字符串
  pdfstartview={FitH},% 文档初始视图为匹配宽度
  bookmarksnumbered={true},% 书签附上章节编号
  bookmarksopen={true},% 展开书签
  pdfborder={0 0 0},% 链接无框
  pdftitle={autobuild tools},
  pdfauthor={descent},
  pdfsubject={develop tools},
  pdfkeywords={autobuild},
  pdfcreator={ps2pdf},
  pdfproducer={PDF 制作程序},% 这个好像没起作用?
}


%\begin{htmlonly}


%\end{htmlonly}

%\renewcommand{\appendixpagenam}{附錄}

\parindent=0cm
\parskip=20pt

% 標題頁
% 正文
\begin{document}
%\renewcommand{\abstractname}{\r18 摘要}
%\renewcommand{\figurename}{\m11 圖}
%\renewcommand{\contentsname}{\r20 目錄}
\renewcommand{\abstractname}{摘要}
\renewcommand{\figurename}{圖}
\renewcommand{\contentsname}{目錄}
\begin{CJK}{UTF8}{bsmi} %開始CJK環境,設定編碼,設定字體

%\begin{htmlonly}
% 標題頁
\begin{titlepage}
\fontsize{30}{30pt plus.5pt minux.4pt}\selectfont
\noindent
%Document Title:\\[6pt]
使用 freetype library 秀出字型

\fontsize{14}{20pt plus.5pt minux.4pt}\selectfont
\par\vfill\noindent

Author: descent\\
%文件版本:$Revision$
%文件版本:$Id$

\today


\end{titlepage}

%\end{htmlonly}

\newpage
%\addtocontents{toc}{text}
\tableofcontents
\addcontentsline{toc}{section}{abstract}
%\addcontentsline{toc}{section}{REFERENCE}
\newpage


\begin{abstract}
本文件介紹如何使用 autoconf, automake, libtool
來產生編譯原始碼所需要的檔案, ex : configure, Makefile ...
\end{abstract}
\newpage

\section{autobuild 簡介}
GNU autobuild 系統包含了 autoconf, automake, libtool。
autoconf 需要編寫 configure.ac,
automake 需要編寫 Makefile.am,

configure.ac 由 M4 巨集和 shell script 構成。
所以在 configure.ac 可以看到 M4 巨集和 shell script。
而 configure.ac 可以先用 autoscan 來產生一個範本 (configure.scan),
先複製成 configure.ac 再修改需要的 M4 巨集即可。

automake 則是用來產生 Makefile.in, 再讓 configure 根據 Makefile.in
產生 Makefile。

libtool 則是產生 static/shared library 的 script。

\section{autobuild 版本}

autoscan (GNU Autoconf) 2.59
Written by David J. MacKenzie and Akim Demaille.

Copyright (C) 2003 Free Software Foundation, Inc.
This is free software; see the source for copying conditions.  There is NO
warranty; not even for MERCHANTABILITY or FITNESS FOR A PARTICULAR PURPOSE.


autoconf (GNU Autoconf) 2.59
Written by David J. MacKenzie and Akim Demaille.

Copyright (C) 2003 Free Software Foundation, Inc.
This is free software; see the source for copying conditions.  There is NO
warranty; not even for MERCHANTABILITY or FITNESS FOR A PARTICULAR PURPOSE.



automake (GNU automake) 1.9.5
Written by Tom Tromey <tromey@redhat.com>.

Copyright 2005 Free Software Foundation, Inc.
This is free software; see the source for copying conditions.  There is NO
warranty; not even for MERCHANTABILITY or FITNESS FOR A PARTICULAR PURPOSE.

由於各版本的 M4 巨集寫法有些差異, 以下介紹以我目前所接觸的最新版本為主,
若是之前的版本不一定適用。

\section{hello example C version}\label{c_hello}

這是 C 版本的 hello。
\begin{lstlisting}[frame=single,frameround=tttt,breaklines=true]
#include <stdio.h>

int main()
{
  printf("hello\n");
  return 0;
}
\end{lstlisting}

\begin{lstlisting}[frame=single,frameround=tttt,breaklines=true]
$ autoscan
autom4te: configure.ac: no such file or directory
autoscan: /usr/bin/autom4te failed with exit status: 1
\end{lstlisting}
configure.scan 內容:

\begin{lstlisting}[frame=single,frameround=tttt,breaklines=true]
# Process this file with autoconf to produce a configure script.

AC_PREREQ(2.59)
AC_INIT(FULL-PACKAGE-NAME, VERSION, BUG-REPORT-ADDRESS)
AC_CONFIG_SRCDIR([hello.c])
AC_CONFIG_HEADER([config.h])

# Checks for programs.
AC_PROG_CC

# Checks for libraries.

# Checks for header files.

# Checks for typedefs, structures, and compiler characteristics.

# Checks for library functions.
AC_OUTPUT
\end{lstlisting}

\begin{verbatim}
$ cp  configure.scan configure.ac
\end{verbatim}

現在我們要修改  configure.ac, 使其符合我們的需要。
首先先移除 \verb+ AC_CONFIG_HEADER([config.h])+,
因為目前我們沒有用到 config.h。

加入和 automake 有關的 M4 巨集。
\begin{verbatim}
AM_INIT_AUTOMAKE
AC_CONFIG_FILES(Makefile)
\end{verbatim}

\verb+AM_INIT_AUTOMAKE+ 初始化 automake。
\verb+AC_CONFIG_FILES(Makefile)+ 會由 Makefile.in 來產生 Makefile,
而 Makefile.in 由 Makefile.am 來產生。


修改 \verb+AC_INIT+ 參數:
\verb+AC_INIT(FULL-PACKAGE-NAME, VERSION, BUG-REPORT-ADDRESS)+
應該很直覺, FULL-PACKAGE-NAME \verb+->+ hello,
VERSION \verb+->+ 0.1,
BUG-REPORT-ADDRESS \verb+->+ \verb+descent@gmail.com+。
這三個參數分別為 package 的 name, 版本, bug report 到那裡。
以下為修改後的 configure.ac:

\begin{lstlisting}[frame=single,frameround=tttt,breaklines=true]
#                                               -*- Autoconf -*-
# Process this file with autoconf to produce a configure script.

AC_PREREQ(2.59)
AC_INIT(hello, 0.1, descent@gmail.com)
AM_INIT_AUTOMAKE
AC_CONFIG_SRCDIR([hello.c])

#AC_CONFIG_HEADER([config.h])

# Checks for programs.
AC_PROG_CC
AC_CONFIG_FILES(Makefile)

# Checks for libraries.

# Checks for header files.

# Checks for typedefs, structures, and compiler characteristics.

# Checks for library functions.
AC_OUTPUT
\end{lstlisting}

再來看看 Makefile.am

\begin{verbatim}
bin_PROGRAMS=hello
hello_SOURCES=hello.c
\end{verbatim}

這個更直覺, 完全不需要解釋了吧!?

再來便是執行一系列的程式來產生所需要的檔案了。

\begin{verbatim}
aclocal
autoconf
automake -a
\end{verbatim}

第一次執行 automake -a 會有以下的訊息:
\begin{lstlisting}[frame=single,frameround=tttt,breaklines=true]
automake: Makefile.am: required file `./NEWS' not found
automake: Makefile.am: required file `./README' not found
automake: Makefile.am: required file `./AUTHORS' not found
automake: Makefile.am: required file `./ChangeLog' not found
\end{lstlisting}
這是因為 automake 需要這些檔案, 而 automake 無法自動產生,
所以我們自己來產生。
\begin{verbatim}
$ touch NEWS README AUTHOR ChangeLog
\end{verbatim}

再執行 automake -a 一次。
應該就沒問題了。

執行 configure script 看看有無錯誤。

\begin{verbatim}
./configure
make
\end{verbatim}
來測試看有沒有正確產生 hello, 並執行看看。

\begin{verbatim}
make distcheck
\end{verbatim}

會作一系列的測試, 並產生 hello-0.1.tar.gz,
hello, 0.1 就是 \verb+AC_INIT+ 的前兩個參數。
可以試試看

\begin{verbatim}
make install
make distclean
\end{verbatim}

這些 makefile rule。 如何?
是不是比自己寫 makfile 容易多了。


\section{hello example C++ version}

這是 C++ 版本的 hello。

\begin{lstlisting}[frame=single,frameround=tttt,breaklines=true]
#include <iostream>

int main()
{
  using namespace std;

  cout << "hello" << endl;
  return 0;
}
\end{lstlisting}

一樣先執行 autoscan, 再把 configure.scan copy 成 configure.ac。
autoscan 會得知目前的語言是 C++ 語言, 並會產生相對應的 M4 巨集。
修改方式和 C 版本的 hello 一樣, 注意 \verb+AC_PROG_CXX+,
這會測試是否有 C++ compiler, 也就是 g++。
照著 section \ref{c_hello} 再執行一遍, 應該可以順利成功。

由於我大部份使用 C++, 所以下面的例子全部以 C++ 為主。


\section{加入 libtool}

\subsection{libtool 簡介}
libtool 是用來產生 library 的 script,
可以用來產生 shared/static library。
由於每個系統產生 library 的方法不同, libtool 把有支援到的平台,
以同樣的方式來產生 library。

下面的命令可以產生 static/shared library。
\begin{verbatim}
libtool --mode=link g++ -o libgraph.la \
svga.lo shape.lo geometry.lo view.lo \
painter.lo draw.lo color_convert.lo \
-rpath /usr/local/lib
\end{verbatim}

而 libtool 命令會去執行下面的 g++ 命令。

\begin{verbatim}
g++ -shared   .libs/svga.o .libs/shape.o .libs/geometry.o \
.libs/view.o .libs/painter.o .libs/draw.o .libs/color_convert.o  \
-lstdc++ -lm -lc -lgcc_s   -Wl,-soname -Wl,libgraph.so.0 \
-o .libs/libgraph.so.0.0.0
(cd .libs && rm -f libgraph.so.0 && ln -s libgraph.so.0.0.0 libgraph.so.0)
(cd .libs && rm -f libgraph.so && ln -s libgraph.so.0.0.0 libgraph.so)
ar cru .libs/libgraph.a  svga.o shape.o geometry.o view.o painter.o \
draw.o color_convert.o
ranlib .libs/libgraph.a
creating libgraph.la
(cd .libs && rm -f libgraph.la && ln -s ../libgraph.la libgraph.la)
$ exit
\end{verbatim}

在 autoconf, automake 加入一些 M4 巨集及變數, 便可以呼叫 libtool 來幫我們產生
library。想要單獨使用 libtool 的朋友, 請自行參考 libtool 手冊。 

\subsection{加入 libtool 支援}
在 configure.ac 新增 \verb+AC_PROG_LIBTOOL+, 就可以了。
至於 Makefile.am:

\begin{verbatim}
bin_PROGRAMS=hello1
hello1_SOURCES=hello1.cpp
hello1_LDADD=libh.la
lib_LTLIBRARIES=libh.la
libh_la_SOURCES=h.cpp h.h
\end{verbatim}

新的變數名稱有:
\begin{verbatim}
hello1_LDADD=libh.la
lib_LTLIBRARIES=libh.la
libh_la_SOURCES=h.cpp h.h
\end{verbatim}

\verb+hello1_LDADD+ 會讓 hello1.o 和 libh.la 作 link,
\verb+lib_LTLIBRARIES+ 會用 libtool 產生 libh.la, 會有 shared/static 兩種,
在 .lib 目錄裡。
\verb+libh_la_SOURCES+ 這個 library 的原始程式。

\verb+libh_la+ 就是將 libh.la 的 . 符號以 \verb+_+ 替換掉,
若是 library 檔名為 libabc.la 就以 \verb+libabc_la_SOURCES+ 為 Makefile.am 的變數名稱。



接著執行
\begin{verbatim}
aclocal
autoconf
libtoolize --force 
automake -a
\end{verbatim}

\verb+libtoolize --force+ 會產生新的 script 檔案,
\begin{verbatim}
config.guess
config.sub
ltmain.sh'
\end{verbatim}

以下是完整的 configure.ac
\begin{verbatim}
#                                               -*- Autoconf -*-
# Process this file with autoconf to produce a configure script.

AC_PREREQ(2.59)
AC_INIT(FULL-PACKAGE-NAME, VERSION, BUG-REPORT-ADDRESS)
AM_INIT_AUTOMAKE
AC_CONFIG_SRCDIR([hello1.cpp])
#AC_CONFIG_HEADER([config.h])

AC_PROG_LIBTOOL

AC_CONFIG_FILES([Makefile])

# Checks for programs.
AC_PROG_CXX

# Checks for libraries.

# Checks for header files.

# Checks for typedefs, structures, and compiler characteristics.

# Checks for library functions.
AC_OUTPUT
\end{verbatim}

以下是完整的 Makefile.am
\begin{verbatim}
bin_PROGRAMS=hello1
hello1_SOURCES=hello1.cpp 
hello1_LDADD=libh.la
lib_LTLIBRARIES=libh.la 
libh_la_SOURCES=h.cpp h.h
\end{verbatim}

所有原始碼檔案:
\begin{verbatim}
h.cpp  hello1.cpp  hello2.cpp  h.h
\end{verbatim}

hello1.cpp hello2.cpp 是兩個含有 main 的程式檔案,
h.h h.cpp 則含有一個 function 來讓 hello1.cpp hello2.cpp 呼叫,
也是構成 libh.la library 的原始程式。

\section{產生兩個以上的執行檔}
這很容易,
把 Makefile.am 改成

\begin{verbatim}
bin_PROGRAMS=hello1 hello2
hello1_SOURCES=hello1.cpp 
hello1_LDADD=libh.la
hello2_SOURCES=hello2.cpp 
hello2_LDADD=libh.la
lib_LTLIBRARIES=libh.la 
libh_la_SOURCES=h.cpp h.h
\end{verbatim}

產生 hello2 相關的變數 (\verb+hello2_SOURCES+, \verb+hello2_LDADD+), \verb+bin_PROGRAMS+ 則加入一個 hello2。以此類推就可以產生多個執行檔。


\section{使用 config.h}


經由 autobuild 產生的 Makefile 會產生很多的 -D 巨集定義,
\verb+AC_CONFIG_HEADER([config.h])+
可以產生 config.h 裡面會定義一些巨集,
而我們的程式


\begin{verbatim}
#if HAVE_CONFIG_H
#include "config.h"
#endif
\end{verbatim}

即可使用這些巨集定義,
而 Makefile 也不會有很多的 -DAAA -DBBB -DCCC,
可以減少 Makefile 的大小。


這是在還沒有使用
\verb+AC_CONFIG_HEADER([config.h])+
所產生的 Makefile 其中的巨集定義。

\begin{verbatim}
DEFS = -DPACKAGE_NAME=\"FULL-PACKAGE-NAME\" -DPACKAGE_TARNAME=\"full-package-name\" -DPACKAGE_VERSION=\"VERSION\" -DPACKAGE_STRING=\"FULL-PACKAGE-NAME\ VERSION\" -DPACKAGE_BUGREPORT=\"BUG-REPORT-ADDRESS\" -DPACKAGE=\"full-package-name\" -DVERSION=\"VERSION\" -DSTDC_HEADERS=1 -DHAVE_SYS_TYPES_H=1 -DHAVE_SYS_STAT_H=1 -DHAVE_STDLIB_H=1 -DHAVE_STRING_H=1 -DHAVE_MEMORY_H=1 -DHAVE_STRINGS_H=1 -DHAVE_INTTYPES_H=1 -DHAVE_STDINT_H=1 -DHAVE_UNISTD_H=1 -DHAVE_DLFCN_H=1
\end{verbatim}

在使用
\verb+AC_CONFIG_HEADER([config.h])+
之後, Makefile 的 DEFS 如下:

\begin{verbatim}
DEFS = -DHAVE_CONFIG_H
\end{verbatim}

所有的巨集定義都在 config.h 裡了。


\begin{verbatim}
/* config.h.  Generated by configure.  */
/* config.h.in.  Generated from configure.ac by autoheader.  */

/* Define to 1 if you have the <dlfcn.h> header file. */
#define HAVE_DLFCN_H 1

/* Define to 1 if you have the <inttypes.h> header file. */
#define HAVE_INTTYPES_H 1

/* Define to 1 if you have the `vga' library (-lvga). */
#define HAVE_LIBVGA 1

/* Define to 1 if you have the `vgagl' library (-lvgagl). */
#define HAVE_LIBVGAGL 1

/* Define to 1 if you have the <memory.h> header file. */
#define HAVE_MEMORY_H 1

/* Define to 1 if you have the <stdint.h> header file. */
#define HAVE_STDINT_H 1

/* Define to 1 if you have the <stdlib.h> header file. */
#define HAVE_STDLIB_H 1

/* Define to 1 if you have the <strings.h> header file. */
#define HAVE_STRINGS_H 1

/* Define to 1 if you have the <string.h> header file. */
#define HAVE_STRING_H 1

/* Define to 1 if you have the <sys/stat.h> header file. */
#define HAVE_SYS_STAT_H 1

/* Define to 1 if you have the <sys/types.h> header file. */
#define HAVE_SYS_TYPES_H 1

/* Define to 1 if you have the <unistd.h> header file. */
#define HAVE_UNISTD_H 1

/* Name of package */
#define PACKAGE "svgalib_test"

/* Define to the address where bug reports for this package should be sent. */
#define PACKAGE_BUGREPORT "descent@gmail.com"

/* Define to the full name of this package. */
#define PACKAGE_NAME "svgalib_test"

/* Define to the full name and version of this package. */
#define PACKAGE_STRING "svgalib_test 0.0.1"

/* Define to the one symbol short name of this package. */
#define PACKAGE_TARNAME "svgalib_test"

/* Define to the version of this package. */
#define PACKAGE_VERSION "0.0.1"

/* Define to 1 if you have the ANSI C header files. */
#define STDC_HEADERS 1

/* Version number of package */
#define VERSION "0.0.1"

/* Define to empty if `const' does not conform to ANSI C. */
/* #undef const */
\end{verbatim}


執行
\begin{verbatim}
aclocal
autoconf
autoheader
libtoolize --force 
automake -a
\end{verbatim}

現在要多執行一支程式: autoheader, 用來產生 config.h。


\section{把原始碼放到 src 的目錄裡}
建立 src 目錄, 將 source code (*.cpp, *.h) 移到 src,
將原本的 Makefile.am 也移到 src,
修改 confiure.ac


\begin{verbatim}
AC_CONFIG_FILES([Makefile src/Makefile])
\end{verbatim}

表示要在原始碼目錄和 src 目錄產生 Makefile,
接著在原始碼目錄新增一個 Makefile.am,
其內容為:

\begin{verbatim}
SUBDIRS = src
\end{verbatim}


執行
\begin{verbatim}
aclocal
autoconf
autoheader
libtoolize --force 
automake -a
\end{verbatim}

執行 configure,
在原始碼目錄可以看見 Makefile, config.h,
而 src 目錄也會產生 Makefile。


\section{撰寫自己的 M4 測試}
以下的例子會檢查 svgalib 和 libjpeg 是否系統有這兩個 library。
\begin{verbatim}

#                                               -*- Autoconf -*-
# Process this file with autoconf to produce a configure script.

AC_PREREQ(2.59)
AC_INIT(jpeglib_test, 0.0.1, descent@gmail.com)
AM_INIT_AUTOMAKE
AC_CONFIG_SRCDIR([jpeg.cpp])
AC_PROG_LIBTOOL
AC_CONFIG_HEADER([config.h])

# Checks for programs.
AC_PROG_CXX
AC_PROG_CC
AC_CONFIG_FILES([Makefile])

# Checks for libraries.
#AC_CHECK_LIB([vga],main,,[echo "not find svgalib vgagl.";exit 1],)
#AC_CHECK_LIB([vgagl],main,,[echo "not find svgalib vga .";exit 1],[-lvga])

AC_LANG_PUSH([C++])
ORG_LIBS="LIBS"
AC_SUBST([CONFIG_SVGALIB_LIB],["-lvga -lvgagl"])
AC_MSG_CHECKING([we need to link svgalib lib])
LIBS="$CONFIG_SVGALIB_LIB $LIBS"
AC_LINK_IFELSE(
[AC_LANG_PROGRAM(
  [[
    #include <vga.h>
    #include <vgagl.h>
  ]],
  [[
    vga_init();
    int mode=G320x200x256;
    vga_setmode(mode);
    gl_setcontextvga(mode);
    vga_setmode(TEXT);
  ]]
)],
[CONFIG_SVGALIB_LIB="-lvga -lvgagl"; AC_MSG_RESULT(yes);],
[AC_MSG_RESULT(no), CONFIG_SVGALIB_LIB="" ; exit;]
)

LIBS=$ORG_LIBS

#check jpeg lib
AC_SUBST([CONFIG_JPEG_LIB],["-ljpeg"])
AC_MSG_CHECKING([we need to link jpeg lib])

LIBS="$CONFIG_JPEG_LIB $LIB"

AC_LINK_IFELSE(
[AC_LANG_PROGRAM(
  [[
    #include <cstdio> // this shoule before jpeg header file
    #include <cstdlib> // this shoule before jpeg header file
    #include <jpeglib.h>
    #include <jerror.h>

  ]],
  [[
    struct jpeg_decompress_struct cinfo;
    struct jpeg_error_mgr jerr;

    cinfo.err = jpeg_std_error (&jerr);

  ]]
)],
[CONFIG_JPEG_LIB="-ljpeg"; AC_MSG_RESULT(yes);],
[AC_MSG_RESULT(no), CONFIG_JPEG_LIB=""; echo "no"; exit;]
)

LIBS=$ORG_LIBS

AC_LANG_POP([C++])

LIBS="$CONFIG_JPEG_LIB $CONFIG_SVGALIB_LIB"

# Checks for header files.

# Checks for typedefs, structures, and compiler characteristics.
AC_C_CONST

# Checks for library functions.
AC_OUTPUT
\end{verbatim}


%\section{支援 cross compile}

\end{CJK}
\end{document}
